% Onepager zu CVE-2020-0601 (Curveball)
% Basierend auf Informationen aus Kryptologie_2_Informationen_PrA.pdf :contentReference[oaicite:0]{index=0}
\documentclass[paper=a4,fontsize=11pt]{scrartcl}
\usepackage[utf8]{inputenc}
\usepackage[T1]{fontenc}
\usepackage[margin=2cm]{geometry}
\usepackage{hyperref}

\begin{document}
\begin{center}
  {\LARGE\bfseries CVE-2020-0601 (Curveball) – Proof-of-Concept}\\[1ex]
  {\small Kurze Zusammenfassung und erweiterter PoC-Überblick}
\end{center}

\section*{Einleitung}
CVE-2020-0601, auch „Curveball“ genannt, ist eine Spoofing-Schwachstelle in der Windows CryptoAPI (Crypt32.dll). Angreifer können manipulierte elliptische Kurvenparameter nutzen, um gefälschte ECC-Zertifikate als vertrauenswürdig zu präsentieren.

\section*{Hintergrund}
Elliptische Kurven-Kryptographie (ECC) verwendet die Struktur von Punkten auf einer elliptischen Kurve über endlichen Feldern, um öffentliche und private Schlüssel zu erzeugen. Windows’ CryptoAPI unterstützt insbesondere die NIST-Standardkurven P-256 (secp256r1) und P-384 (secp384r1).  
\begin{itemize}
  \item \textbf{Kurvenparameter:} Jede Kurve wird durch Parameter \(a, b\), den Primkörper \(p\), einen Basispunkt \(G\) und seine Ordnung \(n\) definiert.  
  \item \textbf{Punkt-Validierung:} Um die Echtheit eines öffentlichen Schlüssels zu prüfen, muss sichergestellt werden, dass der Punkt tatsächlich auf der Kurve liegt und zur richtigen Untergruppe gehört.  
  \item \textbf{Fehlerquelle:} In Crypt32.dll wurde zwar kontrolliert, ob der gegebene „öffentliche Schlüssel“ zum Generator multipliziert werden kann, jedoch entfiel die vollständige Prüfung, ob alle Kurvenparameter – vor allem die Gruppenordnung und die Zugehörigkeit zur Primärkurve – korrekt sind. 
\end{itemize}
\section*{Schwachstelle}
\begin{itemize}
  \item \textbf{Komponente:} Crypt32.dll (Windows 10, Server 2016/2019)
  \item \textbf{Typ:} Fehlerhafte ECC-Parameter- und Punktvalidierung
  \item \textbf{Auswirkung:} Erstellung und Einsatz gefälschter Code-Signing-Zertifikate
\end{itemize}

\section*{Proof-of-Concept (PoC)}
Die PoC-Implementierung umfasst folgende Schritte:
\begin{enumerate}
  \item \textbf{Root-CA-Zertifikat laden:}  
    Herunterladen des legitimen Microsoft ECC-Root-CA-Zertifikats.
  \item \textbf{Spoofed CA-Key ableiten:}  
    Mit einem Python-Skript (\texttt{gen-key.py}) aus den Root-Public-Key-Daten einen neuen Privatschlüssel gewinnen.
  \item \textbf{Gefälschtes CA-Zertifikat generieren:}  
    Erzeugen eines X.509-Zertifikats mit manipulierten Parametern via OpenSSL.
  \item \textbf{Leaf-Zertifikat und CSR erstellen:}  
    Generierung eines frischen ECC-Schlüsselpaares (\texttt{openssl ecparam}) und einer Certificate Signing Request mit v3-Erweiterungen.
  \item \textbf{Signatur mit Spoofed CA:}  
    Signieren des CSR durch das gefälschte CA-Zertifikat (\texttt{openssl x509 -req}), um ein gültig scheinendes Leaf-Zertifikat zu erhalten.
  \item \textbf{PKCS\#12-Bundle und Code Signing:}  
    Export aller Zertifikate und Schlüssel in eine \texttt{.p12}-Datei und Signatur einer Windows-Binärdatei mit \texttt{osslsigncode}.
  \item \textbf{Verifikation:}  
    Prüfung der Signatur unter Windows (Explorer, \texttt{signtool}) – die manipulierte CA erscheint als vertrauenswürdig.
\end{enumerate}

\end{document}
