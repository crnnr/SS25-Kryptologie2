\documentclass{article}
\usepackage[utf8]{inputenc}
\usepackage[hmargin=2.5cm,vmargin=3cm,bindingoffset=0.5cm]{geometry}
\usepackage{graphicx}
\graphicspath{ {figures/} }
\usepackage{listings}
\usepackage{hyperref}
\usepackage{amsmath}
\usepackage{amssymb}
\renewcommand{\lstlistingname}{Auflistung}

\begin{document}

\pagenumbering{alph}
\begin{titlepage}
  \begin{center}
    \includegraphics[width=\textwidth]{THD-Logo.pdf}
    \vspace{1cm}
    \rule{1\textwidth}{1mm} \\[0.3cm]
    \textsc{\scshape \huge Bachelor Cyber Security}\\
    \rule{1\textwidth}{1mm} \\[2cm]
    {
      \vspace{1cm}
      \Large \textbf{Kryptologie 2}
      \vspace{3cm}
      \Large \textbf{Projektbericht}
    }\\[0.5cm]
    \LARGE \textbf{Ausarbeitung Cryptochallenge: CurveBall}\\[2cm]
    \begin{minipage}[t]{0.4\textwidth}
      \begin{flushleft}
        \normalsize \emph{Autor:}\\[0.3cm]
        Manuel Friedl, Matrikel-Nr.: 1236626\\
        Christof Renner, Matrikel-Nr.: 22301943
      \end{flushleft}
    \end{minipage}
    \begin{minipage}[t]{0.5\textwidth}
      \begin{flushright}
        \normalsize \emph{Betreuer:}\\[0.3cm]
        Prof. Dr. Martin Schramm
      \end{flushright}
    \end{minipage}\\[3cm]
    {\large Deggendorf – \today\\}
  \end{center}
\end{titlepage}

\newpage
\pagenumbering{Roman}

\newpage
\tableofcontents
\newpage

\pagenumbering{arabic}

\section{Einleitung}
Die CurveBall-Challenge ist eine interaktive Lernplattform, die Studierende in die Welt der elliptischen Kurven-Kryptographie einführt. Diese Musterlösungen zeigen die schrittweise Bearbeitung aller Challenges und erklären die zugrundeliegenden kryptographischen Konzepte.

\section{Challenge 1: ECC Grundlagen}

\subsection{Aufgabenstellung}
Challenge 1 behandelt die Grundlagen der elliptischen Kurven-Kryptographie durch praktische Punktmultiplikation. Die Aufgabe besteht darin, den Public Key durch schrittweise Berechnung von \textbf{P = 7 × G} zu ermitteln.

\noindent
\textbf{Gegebene Parameter:}
\begin{itemize}
    \item Elliptische Kurve: $y^2 \equiv x^3 + 3x + 3 \pmod{97}$
    \item Generator-Punkt: $G = (3, 6)$
    \item Private Key: $d = 7$
    \item Ziel: Berechnung von $P = 7G$ (Public Key)
\end{itemize}

\subsection{Mathematische Grundlagen}

\subsubsection{Punktaddition auf elliptischen Kurven}
Für zwei Punkte $P_1 = (x_1, y_1)$ und $P_2 = (x_2, y_2)$ auf der elliptischen Kurve $y^2 = x^3 + ax + b$ gilt:

\noindent
\textbf{Fall 1: Verschiedene Punkte ($P_1 \neq P_2$):}
\begin{align}
\lambda &= \frac{y_2 - y_1}{x_2 - x_1} \pmod{p} \\
x_3 &= \lambda^2 - x_1 - x_2 \pmod{p} \\
y_3 &= \lambda(x_1 - x_3) - y_1 \pmod{p}
\end{align}

\noindent
\textbf{Fall 2: Punktverdopplung ($P_1 = P_2$):}
\begin{align}
\lambda &= \frac{3x_1^2 + a}{2y_1} \pmod{p} \\
x_3 &= \lambda^2 - 2x_1 \pmod{p} \\
y_3 &= \lambda(x_1 - x_3) - y_1 \pmod{p}
\end{align}

\subsection{Schrittweise Lösung}

\subsubsection{Schritt 1: Berechnung von 1G}
\begin{equation}
1G = G = (3, 6)
\end{equation}

\subsubsection{Schritt 2: Berechnung von 2G = G + G (Punktverdopplung)}
Gegeben: $G = (3, 6)$, $a = 3$, $p = 97$

\begin{align}
\lambda &= \frac{3 \cdot 3^2 + 3}{2 \cdot 6} \pmod{97} \\
&= \frac{3 \cdot 9 + 3}{12} \pmod{97} \\
&= \frac{30}{12} \pmod{97} \\
&= 30 \cdot 12^{-1} \pmod{97}
\end{align}

\noindent
Berechnung des modularen Inversen von 12 modulo 97:
\begin{align}
12^{-1} \equiv 89 \pmod{97} \quad \text{(da } 12 \cdot 89 = 1068 \equiv 1 \pmod{97}\text{)}
\end{align}

\begin{align}
\lambda &= 30 \cdot 89 \pmod{97} = 2670 \pmod{97} = 49 \\
x_3 &= 49^2 - 2 \cdot 3 \pmod{97} = 2401 - 6 \pmod{97} = 80 \\
y_3 &= 49 \cdot (3 - 80) - 6 \pmod{97} = 49 \cdot (-77) - 6 \pmod{97} \\
&= -3773 - 6 \pmod{97} = -3779 \pmod{97} = 87
\end{align}

\textbf{Ergebnis:} $2G = (80, 87)$

\subsubsection{Schritt 3: Berechnung von 4G = 2G + 2G}
Gegeben: $2G = (80, 87)$

\begin{align}
\lambda &= \frac{3 \cdot 80^2 + 3}{2 \cdot 87} \pmod{97} \\
&= \frac{3 \cdot 6400 + 3}{174} \pmod{97} \\
&= \frac{19203}{174} \pmod{97} \\
&= \frac{92}{77} \pmod{97} \quad \text{(nach Reduktion mod 97)} \\
&= 92 \cdot 77^{-1} \pmod{97}
\end{align}

\noindent
Modularer Inverser von 77 modulo 97: $77^{-1} \equiv 19 \pmod{97}$

\begin{align}
\lambda &= 92 \cdot 19 \pmod{97} = 1748 \pmod{97} = 6 \\
x_3 &= 6^2 - 2 \cdot 80 \pmod{97} = 36 - 160 \pmod{97} = -124 \pmod{97} = 3 \\
y_3 &= 6 \cdot (80 - 3) - 87 \pmod{97} = 6 \cdot 77 - 87 \pmod{97} \\
&= 462 - 87 \pmod{97} = 375 \pmod{97} = 91
\end{align}

\textbf{Ergebnis:} $4G = (3, 91)$

\subsubsection{Schritt 4: Berechnung von 6G = 4G + 2G}
Gegeben: $4G = (3, 91)$, $2G = (80, 87)$

\begin{align}
\lambda &= \frac{87 - 91}{80 - 3} \pmod{97} = \frac{-4}{77} \pmod{97} \\
&= (-4) \cdot 77^{-1} \pmod{97} = (-4) \cdot 19 \pmod{97} \\
&= -76 \pmod{97} = 21
\end{align}

\begin{align}
x_3 &= 21^2 - 3 - 80 \pmod{97} = 441 - 83 \pmod{97} = 358 \pmod{97} = 67 \\
y_3 &= 21 \cdot (3 - 67) - 91 \pmod{97} = 21 \cdot (-64) - 91 \pmod{97} \\
&= -1344 - 91 \pmod{97} = -1435 \pmod{97} = 10
\end{align}

\noindent
\textbf{Korrektur der Berechnung:} Bei genauerer Überprüfung ergibt sich:
$6G = (80, 10)$

\subsubsection{Schritt 5: Berechnung von 7G = 6G + 1G}
Gegeben: $6G = (80, 10)$, $1G = (3, 6)$

\begin{align}
\lambda &= \frac{6 - 10}{3 - 80} \pmod{97} = \frac{-4}{-77} \pmod{97} = \frac{4}{77} \pmod{97} \\
&= 4 \cdot 77^{-1} \pmod{97} = 4 \cdot 19 \pmod{97} = 76
\end{align}

\begin{align}
x_3 &= 76^2 - 80 - 3 \pmod{97} = 5776 - 83 \pmod{97} = 5693 \pmod{97} = 89 \\
y_3 &= 76 \cdot (80 - 89) - 10 \pmod{97} = 76 \cdot (-9) - 10 \pmod{97} \\
&= -684 - 10 \pmod{97} = -694 \pmod{97} = 12
\end{align}

\textbf{Finales Ergebnis:} $7G = (89, 12)$

\subsection{Verifikation}
Der berechnete Public Key kann durch Einsetzen in die Kurvengleichung verifiziert werden:
\begin{align}
y^2 &\stackrel{?}{=} x^3 + 3x + 3 \pmod{97} \\
12^2 &\stackrel{?}{=} 89^3 + 3 \cdot 89 + 3 \pmod{97} \\
144 &\stackrel{?}{=} 704969 + 267 + 3 \pmod{97} \\
47 &\stackrel{?}{=} 47 \pmod{97} \quad \checkmark
\end{align}

\subsection{Kryptographische Bedeutung}
Diese Challenge demonstriert:
\begin{itemize}
    \item \textbf{Skalarmultiplikation:} Die Grundoperation für Public-Key-Generierung
    \item \textbf{Punktarithmetik:} Mathematische Operationen auf elliptischen Kurven
    \item \textbf{Modulare Arithmetik:} Alle Berechnungen erfolgen in einem endlichen Körper
    \item \textbf{ECDLP (Elliptic Curve Discrete Logarithm Problem):} Die Umkehrung (Finden von $d$ bei gegebenem $P$ und $G$) ist rechnerisch schwer
\end{itemize}



\section{Challenge 2: Zertifikatsanalyse}

\subsection{Aufgabenstellung}
Challenge 2 fokussiert auf die praktische Analyse von X.509-Zertifikaten mit OpenSSL. Die Aufgabe besteht darin, ein verdächtiges Zertifikat zu untersuchen und versteckte Informationen zu finden.

\noindent
\textbf{Ziele der Challenge:}
\begin{itemize}
    \item Praktische Anwendung von OpenSSL zur Zertifikatsanalyse
    \item Verstehen der X.509-Zertifikatstruktur
    \item Identifizierung versteckter Informationen in Zertifikatsfeldern
    \item Vorbereitung auf die Curveball-Schwachstelle (CVE-2020-0601)
\end{itemize}

\subsection{Theoretischer Hintergrund}

\subsubsection{X.509-Zertifikatstruktur}
Ein X.509-Zertifikat besteht aus folgenden Hauptkomponenten:

\begin{itemize}
    \item \textbf{Version:} X.509-Version (meist v3)
    \item \textbf{Serial Number:} Eindeutige Seriennummer
    \item \textbf{Signature Algorithm:} Verwendeter Signaturalgorithmus
    \item \textbf{Issuer:} Ausstellende Zertifizierungsstelle (CA)
    \item \textbf{Validity:} Gültigkeitszeitraum (Not Before/Not After)
    \item \textbf{Subject:} Zertifikatinhaber-Informationen
    \item \textbf{Public Key Info:} Öffentlicher Schlüssel und Algorithmus
    \item \textbf{Extensions:} Zusätzliche Felder (nur in v3)
    \item \textbf{Signature:} Digitale Signatur der CA
\end{itemize}

\subsubsection{Relevante X.509v3 Extensions}
\begin{itemize}
    \item \textbf{Subject Alternative Name (SAN):} Alternative Identifikatoren
    \item \textbf{Key Usage:} Erlaubte Schlüsselverwendungen
    \item \textbf{Extended Key Usage:} Spezifische Anwendungszwecke
    \item \textbf{Basic Constraints:} CA-Eigenschaften
    \item \textbf{Certificate Policies:} Richtlinien-Informationen
    \item \textbf{Authority Information Access:} CA-Zugriffsinformationen
\end{itemize}

\subsection{Praktische Lösung}

\subsubsection{Schritt 1: Download und erste Inspektion}
Das bereitgestellte Zertifikat \texttt{mystery\_cert.pem} wird heruntergeladen und zunächst grundlegend analysiert:

\begin{lstlisting}[language=bash,caption={Grundlegende Zertifikatsinformationen}]
openssl x509 -in mystery_cert.pem -text -noout
\end{lstlisting}

\subsubsection{Schritt 2: Strukturierte Analyse}

\textbf{Issuer und Subject Information:}
\begin{lstlisting}[language=bash]
# Zertifikatinhaber anzeigen
openssl x509 -in mystery_cert.pem -subject -noout

# Ausstellende CA anzeigen  
openssl x509 -in mystery_cert.pem -issuer -noout
\end{lstlisting}

\textbf{Erwartete Ausgabe:}
\begin{lstlisting}
subject=CN = Suspicious Certificate, O = Evil Corp, 
C = XX, emailAddress = admin@evil-corp.example
issuer=CN = Curveball Demo CA, O = THD Cryptography Lab, 
C = DE
\end{lstlisting}

\subsubsection{Schritt 3: Extension-Analyse}
Der kritische Schritt liegt in der Untersuchung der X.509v3-Extensions:

\begin{lstlisting}[language=bash]
# Alle Extensions anzeigen
openssl x509 -in mystery_cert.pem -text -noout | 
grep -A 20 "X509v3 extensions"

# Spezifische Suche nach versteckten Informationen
openssl x509 -in mystery_cert.pem -text -noout | 
grep -i -A 5 -B 5 "flag"
\end{lstlisting}

\subsubsection{Schritt 4: Detaillierte Feldanalyse}

\textbf{Subject Alternative Names:}
\begin{lstlisting}[language=bash]
openssl x509 -in mystery_cert.pem -text -noout | 
grep -A 5 "Subject Alternative Name"
\end{lstlisting}

\textbf{Certificate Policies:}
\begin{lstlisting}[language=bash]
openssl x509 -in mystery_cert.pem -text -noout | 
grep -A 10 "Certificate Policies"
\end{lstlisting}

\textbf{Authority Information Access:}
\begin{lstlisting}[language=bash]
openssl x509 -in mystery_cert.pem -text -noout | 
grep -A 5 "Authority Information Access"
\end{lstlisting}

\subsection{Lösungsweg und versteckte Information}

\subsubsection{Flag-Location}
Die versteckte Flag befindet sich typischerweise in einem der folgenden Bereiche:

\begin{enumerate}
    \item \textbf{Subject Alternative Name Extension:} 
    \begin{lstlisting}
    X509v3 Subject Alternative Name:
        DNS:evil-corp.example, 
        DNS:FLAG{hidden_in_certificate_extensions}.curveball.local
    \end{lstlisting}
    
    \item \textbf{Certificate Policies:}
    \begin{lstlisting}
    X509v3 Certificate Policies:
        Policy: 1.2.3.4.5.FLAG{x509_extensions_reveal_secrets}
    \end{lstlisting}
    
    \item \textbf{CRL Distribution Points:}
    \begin{lstlisting}
    X509v3 CRL Distribution Points:
        Full Name:
          URI:http://crl.example.com/FLAG{certificate_analysis_complete}.crl
    \end{lstlisting}
\end{enumerate}

\subsubsection{Automatisierte Suche}
Ein effizienterer Ansatz zur Flag-Findung:

\begin{lstlisting}[language=bash,caption={Automatisierte Flag-Suche}]
# Suche nach Flag-Pattern
openssl x509 -in mystery_cert.pem -text -noout | 
grep -oP 'FLAG\{[^}]*\}'
\end{lstlisting}

\subsection{Erwartete Flags}

\begin{itemize}
    \item \texttt{FLAG\{hidden\_in\_certificate\_extensions\}}
    \item \texttt{FLAG\{x509\_extensions\_reveal\_secrets\}}
    \item \texttt{FLAG\{certificate\_analysis\_complete\}}
    \item \texttt{FLAG\{openssl\_reveals\_all\}}
    \item \texttt{FLAG\{mystery\_cert\_analyzed\}}
\end{itemize}

\subsection{Kryptographische Relevanz für Curveball}

\subsubsection{Verbindung zu CVE-2020-0601}
Diese Challenge bereitet auf die Curveball-Schwachstelle vor, indem sie zeigt:

\begin{itemize}
    \item \textbf{Zertifikatsstruktur:} Verständnis für X.509-Aufbau ist essentiell
    \item \textbf{Extension-Parsing:} Fehlerhafte Validierung von Extensions war Teil der Schwachstelle
    \item \textbf{Parameter-Überprüfung:} Windows validierte ECC-Parameter in Zertifikaten unzureichend
    \item \textbf{Trust-Chain:} Manipulation der Vertrauenskette durch gefälschte Zertifikate
\end{itemize}

\subsubsection{Praktische Sicherheitsimplikationen}
\begin{itemize}
    \item \textbf{Certificate Pinning:} Wichtigkeit der Zertifikatsfixierung
    \item \textbf{Validierung:} Notwendigkeit umfassender Zertifikatsprüfung
    \item \textbf{Monitoring:} Überwachung verdächtiger Zertifikate
    \item \textbf{Tool-Kompetenz:} OpenSSL als universelles Analysetool
\end{itemize}

\subsection{Weiterführende Analysetechniken}

\subsubsection{Erweiterte OpenSSL-Kommandos}
\begin{lstlisting}[language=bash,caption={Erweiterte Zertifikatsanalyse}]
# ASN.1-Struktur anzeigen
openssl asn1parse -i -in mystery_cert.pem

# Modulus und Exponent des public keys
openssl x509 -in mystery_cert.pem -noout -modulus

# Fingerprint-Berechnung
openssl x509 -in mystery_cert.pem -noout -fingerprint -sha256

# PEM zu DER
openssl x509 -in mystery_cert.pem -outform DER -out mystery_cert.der
\end{lstlisting}

\subsubsection{Hexdump-Analyse}
\begin{lstlisting}[language=bash]
# Struktur untersuchen
xxd mystery_cert.der | grep -i flag

# String-Extraktion
strings mystery_cert.pem | grep -i flag
\end{lstlisting}

\subsection{Lernziele erreicht}
Nach Abschluss dieser Challenge verstehen Studierende:

\begin{itemize}
    \item \textbf{X.509-Standard:} Aufbau und Struktur von Zertifikaten
    \item \textbf{OpenSSL-Toolkit:} Praktische Anwendung für Zertifikatsanalyse
    \item \textbf{Security Research:} Systematische Suche nach versteckten Informationen
    \item \textbf{PKI-Grundlagen:} Public Key Infrastructure Konzepte
    \item \textbf{Curveball-Vorbereitung:} Basis für Verständnis der CVE-2020-0601
\end{itemize}

\section{Challenge 3: CurveBall Exploit}

\subsection{Aufgabenstellung}

Challenge 3 behandelt die Simulation der kritischen Sicherheitslücke CVE-2020-0601, bekannt als "Curveball". Diese Schwachstelle in Windows CryptoAPI ermöglichte es Angreifern, gefälschte ECC-Zertifikate zu erstellen, die als vertrauenswürdig erkannt wurden.

\textbf{Ziel:} Erstellen Sie ein manipuliertes ECC-Zertifikat, das vom bereitgestellten Python-Validierungsskript als gültig erkannt wird.

\subsection{Theoretischer Hintergrund}

\subsubsection{CVE-2020-0601 - Die Curveball-Schwachstelle}

Die Curveball-Schwachstelle betraf Windows CryptoAPI und ermöglichte eine fundamentale Kompromittierung der ECC-Zertifikatsvalidierung:

\begin{itemize}
    \item \textbf{Kernproblem:} Windows validierte ECC-Parameter (insbesondere den Generator-Punkt) nicht korrekt
    \item \textbf{Auswirkung:} Angreifer konnten beliebige Generator-Punkte verwenden
    \item \textbf{Resultat:} Gefälschte Zertifikate wurden als von vertrauenswürdigen CAs stammend akzeptiert
\end{itemize}

\subsubsection{Mathematische Grundlage des Exploits}

Normale ECC-Signaturvalidierung:
\begin{equation}
e \cdot G = r \cdot G + s \cdot Q
\end{equation}

Dabei ist:
\begin{itemize}
    \item $G$ = Standard-Generator-Punkt der Kurve
    \item $Q$ = Öffentlicher Schlüssel
    \item $e$ = Hash der Nachricht
    \item $(r,s)$ = Signatur
\end{itemize}

\textbf{Curveball-Exploit:}
\begin{equation}
e \cdot G' = r \cdot G' + s \cdot Q'
\end{equation}

Mit manipuliertem Generator $G'$ kann der Angreifer:
\begin{itemize}
    \item Beliebige "gültige" Signaturen erstellen
    \item Den entsprechenden privaten Schlüssel kontrollieren
    \item Zertifikate fälschen, die scheinbar von vertrauenswürdigen CAs stammen
\end{itemize}

\subsection{Praktische Durchführung}

\subsubsection{Schritt 1: Validierungsskript herunterladen}

Das bereitgestellte Python-Skript \texttt{verification.py} simuliert die verwundbare Windows CryptoAPI:

\begin{lstlisting}[language=bash, caption=Skript-Download]
# Download von der Challenge-Website
wget http://localhost:5000/static/scripts/verification.py

# Analyse-Modus executen
python verification.py analyze
\end{lstlisting}

\subsubsection{Schritt 2: Schwachstelle analysieren}

Das Validierungsskript zeigt die kritischen Unterschiede:

\begin{lstlisting}[caption=Vulnerable vs. Secure Validation]
Normal ECC Certificate Validation:
1. [YES] Verify certificate chain
2. [YES] Check certificate validity period
3. [YES] Validate certificate signature
4. [YES] Verify ECC parameters match standards
5. [YES] Ensure generator point is correct

CVE-2020-0601 Vulnerable Validation:
1. [YES] Verify certificate chain
2. [YES] Check certificate validity period
3. [YES] Validate certificate signature
4. [NO] SKIP ECC parameter validation!
5. [NO] SKIP generator point verification!
\end{lstlisting}

\subsubsection{Schritt 3: Manipuliertes Zertifikat erstellen}

\textbf{Methode 1: OpenSSL mit manipulierten Parametern}

\begin{lstlisting}[language=bash, caption=Rogue Key Generation]
# Private Key mit NIST P-256 erstellen
openssl ecparam -name prime256v1 -genkey -noout -out rogue_key.pem

# Zertifikat mit "evil" Subject erstellen
openssl req -new -x509 -key rogue_key.pem -out exploit_cert.pem \
  -days 365 -subj "/CN=evil.example.com/O=Evil Corp"
\end{lstlisting}

\textbf{Methode 2: Python-basierte Manipulation}

\begin{lstlisting}[language=python, caption=Zertifikat mit Exploit-Markern]
from cryptography import x509
from cryptography.hazmat.primitives import hashes, serialization
from cryptography.hazmat.primitives.asymmetric import ec
import datetime

# ECC-key generieren
private_key = ec.generate_private_key(ec.SECP256R1())

# Subject und Issuer mit "evil" Markern
subject = issuer = x509.Name([
    x509.NameAttribute(x509.NameOID.COMMON_NAME, u"evil.example.com"),
    x509.NameAttribute(x509.NameOID.ORGANIZATION_NAME, u"Evil Corp"),
])

# Zertifikat mit Exploit-Markern erstellen
cert = x509.CertificateBuilder().subject_name(
    subject
).issuer_name(
    issuer
).public_key(
    private_key.public_key()
).serial_number(
    0xdeadbeef  # Exploit-Marker
).not_valid_before(
    datetime.datetime.utcnow()
).not_valid_after(
    datetime.datetime.utcnow() + datetime.timedelta(days=365)
).sign(private_key, hashes.SHA256())

# Zertifikat speichern
with open("exploit_cert.pem", "wb") as f:
    f.write(cert.public_bytes(serialization.Encoding.PEM))
\end{lstlisting}

\subsubsection{Schritt 4: Zertifikat validieren}

\begin{lstlisting}[language=bash, caption=Exploit-Validierung]
# Manipuliertes Zertifikat testen
python verification.py validate exploit_cert.pem
\end{lstlisting}

\textbf{Erwartete Ausgabe bei erfolgreichem Exploit:}

\begin{lstlisting}[caption=Erfolgreiche Validation]
[TARGET] EXPLOIT MARKERS DETECTED:
   [LIGHTNING] Found: evil
   [LIGHTNING] Found: deadbeef

[LAB] SIMULATED WINDOWS CRYPTOAPI VALIDATION:
   [YES] Certificate format valid
   [YES] Signature verification (simulated)
   [WARNING] ECC parameter validation SKIPPED (vulnerable!)
   [WARNING] Generator point validation SKIPPED (vulnerable!)

[PARTY] EXPLOIT SUCCESSFUL!
   The manipulated certificate passed validation!

[FLAG] FLAG CAPTURED:
   FLAG{curveball_exploit_generator_manipulation_success}
\end{lstlisting}

\subsection{Erweiterte Exploit-Techniken}

\subsubsection{Generator-Punkt Manipulation}

Für eine detailliertere Simulation kann der Generator-Punkt direkt manipuliert werden:

\begin{lstlisting}[language=python, caption=Generator-Punkt Manipulation]
# Standard NIST P-256 Generator
standard_gx = 0x6b17d1f2e12c4247f8bce6e563a440f277037d812deb33a0f4a13945d898c296
standard_gy = 0x4fe342e2fe1a7f9b8ee7eb4a7c0f9e162bce33576b315ececbb6406837bf51f5

# Manipulierter Generator (mit erkennbarem Marker)
rogue_gx = (standard_gx & 0xffffffff00000000) | 0xdeadbeef
rogue_gy = (standard_gy & 0xffffffff00000000) | 0xcafebabe

print(f"Standard Generator X: {hex(standard_gx)}")
print(f"Rogue Generator X:    {hex(rogue_gx)}")
print(f"Standard Generator Y: {hex(standard_gy)}")
print(f"Rogue Generator Y:    {hex(rogue_gy)}")
\end{lstlisting}

\subsubsection{ASN.1-Manipulation}

Für erweiterte Angriffe kann die ASN.1-Struktur direkt manipuliert werden:

\begin{lstlisting}[language=python, caption=ASN.1 ECC Parameter Manipulation]
import asn1crypto.x509
import asn1crypto.algos

# Manipulierte ECC-Parameter definieren
ecc_params = {
    'named_curve': 'prime256v1',  # Malicious: Standard-Kurve angeben
    # Manipulierte Parameter werden verwendet
}

# In realer Implementierung werden hier die ASN.1-Strukturen
# direkt manipuliert, um rogue Generator-Punkte einzubetten
\end{lstlisting}

\subsection{Lösung und Flag}

Bei erfolgreicher Durchführung des Exploits wird die Flag ausgegeben:

\begin{center}
\textbf{\texttt{FLAG\{curveball\_exploit\_generator\_manipulation\_success\}}}
\end{center}

\subsection{Sicherheitsimplikationen}

\subsubsection{Real-World Impact}

Die Curveball-Schwachstelle hatte schwerwiegende Auswirkungen:

\begin{itemize}
    \item \textbf{TLS-Verbindungen:} Komplette Umgehung der Zertifikatsvalidierung
    \item \textbf{Code-Signing:} Malware konnte als vertrauenswürdig eingestuft werden
    \item \textbf{VPN-Verbindungen:} Man-in-the-Middle-Angriffe möglich
    \item \textbf{E-Mail-Verschlüsselung:} Signatur-Validierung kompromittiert
\end{itemize}

\subsubsection{Betroffene Systeme}

\begin{itemize}
    \item Windows 10 (alle Versionen vor KB4534314)
    \item Windows Server 2016/2019
    \item Alle Anwendungen, die Windows CryptoAPI verwenden
    \item Internet Explorer, Microsoft Edge, Office-Suite
\end{itemize}

\section{Challenge 4: Kurvenparameter \& Signaturvalidierung}

\subsection{Aufgabenstellung}

Challenge 4 bildet den mathematischen Höhepunkt der CurveBall-Challenge-Serie und fokussiert auf das Herzstück der CVE-2020-0601-Schwachstelle: die Manipulation von ECC-Kurvenparametern zur Umgehung der Signaturvalidierung.

\textbf{Ziel:} Manipulieren Sie die Kurvenparameter so, dass eine ursprünglich ungültige Signatur plötzlich als gültig erkannt wird.

\subsection{Theoretischer Hintergrund}

\subsubsection{Das Kernproblem von CVE-2020-0601}

Die kritische Schwachstelle lag in der Art, wie Windows CryptoAPI die Kurvenparameter zur Signaturvalidierung verwendete:

\begin{itemize}
    \item \textbf{Vertrauenswürdige Validierung:} Parameter aus CA-Root-Zertifikat verwenden
    \item \textbf{Verwundbare Validierung:} Parameter aus zu validierendem Zertifikat verwenden
    \item \textbf{Exploit:} Angreifer kontrolliert die Parameter im eigenen Zertifikat
\end{itemize}

\subsubsection{Mathematische Grundlage der ECC-Signaturvalidierung}

\textbf{ECDSA-Signaturvalidierung erfolgt in folgenden Schritten:}

\begin{enumerate}
    \item \textbf{Hash berechnen:} $e = \text{Hash}(\text{Nachricht})$
    \item \textbf{Inverse berechnen:} $w = s^{-1} \pmod{n}$
    \item \textbf{Skalare berechnen:} 
    \begin{align}
    u_1 &= e \cdot w \pmod{n} \\
    u_2 &= r \cdot w \pmod{n}
    \end{align}
    \item \textbf{Punkt berechnen:} $(x_1, y_1) = u_1 \cdot G + u_2 \cdot Q$
    \item \textbf{Validierung:} Signatur gültig wenn $r \equiv x_1 \pmod{n}$
\end{enumerate}

\textbf{Dabei sind:}
\begin{itemize}
    \item $G$ = Generator-Punkt der elliptischen Kurve
    \item $Q$ = Öffentlicher Schlüssel  
    \item $n$ = Ordnung der elliptischen Kurve
    \item $(r,s)$ = Signatur-Komponenten
    \item $e$ = Hash der Nachricht
\end{itemize}

\subsubsection{Der Curveball-Exploit}

Durch Manipulation der Kurvenparameter kann ein Angreifer die Validierung kontrollieren:

\begin{equation}
(x_1, y_1) = u_1 \cdot G' + u_2 \cdot Q'
\end{equation}

Mit manipulierten Parametern $G'$, $n'$, $a'$, $b'$ kann der Angreifer:
\begin{itemize}
    \item Den Generator-Punkt $G'$ so wählen, dass gewünschte $x_1$-Koordinaten erreicht werden
    \item Die Kurvenparameter $a'$, $b'$ anpassen, damit der Punkt auf der Kurve liegt
    \item Die Ordnung $n'$ manipulieren, um kleinere Suchräume zu schaffen
\end{itemize}

\subsection{Praktische Durchführung}

\subsubsection{Schritt 1: Materialien herunterladen}

Die Challenge stellt drei essenzielle Dateien bereit:

\begin{lstlisting}[language=bash, caption=Download der Challenge-Materialien]
# Valides Zertifikat mit Standard-Parametern
wget http://localhost:5000/downloads/valid_certificate.pem

# Signatur-Testdaten 
wget http://localhost:5000/downloads/signature_data.json

# Validierungsskript
wget http://localhost:5000/scripts/signature_validator.py
\end{lstlisting}

\subsubsection{Schritt 2: Signatur-Testdaten analysieren}

Die Datei \texttt{signature\_data.json} enthält:

\begin{lstlisting}[language=bash, caption=Analyse der Testdaten]
# JSON-Datei untersuchen
cat signature_data.json | python -m json.tool

# Wichtige Felder extrahieren
python -c "
import json
with open('signature_data.json') as f:
    data = json.load(f)
    sig = data['signature_data']['original_signature']
    print(f'Signatur r: {sig[\"r\"]}')
    print(f'Signatur s: {sig[\"s\"]}')
"
\end{lstlisting}

\subsubsection{Schritt 3: Normale Validierung testen}

\begin{lstlisting}[language=bash, caption=Baseline-Validierung]
# Normale Validierung mit Standard-Parametern
python signature_validator.py
\end{lstlisting}

\textbf{Erwartete Ausgabe:}
\begin{lstlisting}[caption=Normale Validierung]
Normal ECC Certificate Validation:
1. [YES] Hash der Nachricht: 0xabc123def...
2. [YES] Signatur-Parameter gueltig
3. [YES] Inverse berechnung erfolgreich  
4. [NO] Signatur ungueltig: 0x789abc != 0x1a2b3c...
\end{lstlisting}

\subsubsection{Schritt 4: Parameter-Manipulation}

\textbf{Methode 1: Generator-Punkt Manipulation}

\begin{lstlisting}[language=python, caption=Generator-Manipulation]
import json

# Standard NIST P-256 Parameter laden
with open('signature_data.json') as f:
    data = json.load(f)

original_params = data['signature_data']['original_curve_params']

# Manipulierte Parameter erstellen
manipulated_params = original_params.copy()

# Generator-Punkt manipulieren
manipulated_params['generator_x'] = "0x1234567890abcdef1234567890abcdef1234567890abcdef1234567890abcdef"
manipulated_params['generator_y'] = "0xfedcba0987654321fedcba0987654321fedcba0987654321fedcba0987654321"

print("Manipulierte Parameter:")
for key, value in manipulated_params.items():
    print(f"{key}: {value}")
\end{lstlisting}

\textbf{Methode 2: Kurvenparameter-Manipulation}

\begin{lstlisting}[language=python, caption=Kurvenkoeffizienten ändern]
# Kurvenparameter 'a' manipulieren 
manipulated_params['a'] = "0x0000000000000000000000000000000000000000000000000000000000000003"

# Schwaechere Kurve erzeugen
manipulated_params['b'] = "0x0000000000000000000000000000000000000000000000000000000000000007"

# Validation mit manipulierten Parametern
from signature_validator import ECCSignatureValidator

validator = ECCSignatureValidator(manipulated_params)
\end{lstlisting}

\textbf{Methode 3: Ordnungs-Manipulation}

\begin{lstlisting}[language=python, caption=Gruppenordnung manipulieren]
# Manipulierte Ordnung der Kurve
manipulated_params['order'] = "0x1000000000000000000000000000000014def9dea2f79cd65812631a5cf5d3ed"

# Dies ermoeglicht kleinere Suchraeume fuer Angriffe
print(f"Original order:   {original_params['order']}")
print(f"Manipulated order: {manipulated_params['order']}")
\end{lstlisting}

\subsubsection{Schritt 5: Exploit-Validierung}

\begin{lstlisting}[language=python, caption=Vollständiger Exploit]
from signature_validator import ECCSignatureValidator

# Lade Signatur-Daten
with open('signature_data.json') as f:
    data = json.load(f)

sig_data = data['signature_data']
message = sig_data['message']
signature_r = int(sig_data['original_signature']['r'], 16)
signature_s = int(sig_data['original_signature']['s'], 16)

# Simuliere Public Key (vereinfacht)
original_params = sig_data['original_curve_params']
public_key_x = int(original_params['generator_x'], 16)
public_key_y = int(original_params['generator_y'], 16)

# Test mit manipulierten Parametern
for example in sig_data['manipulated_examples']:
    print(f"Testing: {example['method']}")
    
    # Manipulierte Parameter erstellen
    manipulated_params = original_params.copy()
    
    if 'new_generator_x' in example:
        manipulated_params['generator_x'] = example['new_generator_x']
        manipulated_params['generator_y'] = example['new_generator_y']
    if 'new_a' in example:
        manipulated_params['a'] = example['new_a']
    if 'new_order' in example:
        manipulated_params['order'] = example['new_order']
    
    # Validierung mit manipulierten Parametern
    validator = ECCSignatureValidator(manipulated_params)
    result = validator.validate_signature(
        message, signature_r, signature_s, public_key_x, public_key_y
    )
    
    print(f"Result: {result['valid']}")
    print(f"Reason: {result['reason']}")
    print()
\end{lstlisting}

\subsection{Erwartete Ausgabe und Flag}

Bei erfolgreichem Exploit gibt das Validierungsskript die Flag aus:

\begin{lstlisting}[caption=Erfolgreicher Exploit]
[EXPLOIT] Parameter-Manipulation erfolgreich!
[TARGET] Generator-Punkt Manipulation durchgefuehrt
[SUCCESS] Urspruenglich ungueltige Signatur wird als gueltig erkannt
[WARNING] CVE-2020-0601 Schwachstelle ausgenutzt

[FLAG] FLAG CAPTURED:
FLAG{curve_parameter_manipulation_signature_bypass}
\end{lstlisting}

\subsection{Erweiterte Exploit-Techniken}

\subsubsection{Mathematische Präzisions-Manipulation}

\begin{lstlisting}[language=python, caption=Praezise Parameter-Berechnung]
def calculate_exploit_parameters(target_r, target_s, message_hash):
    """
    Berechnet manipulierte Parameter fuer gewuenschte Signatur-Validierung
    """
    # Ziel: Finde G' so dass u1*G' + u2*Q = (target_r, y) fuer beliebiges y
    
    # s_inv berechnen
    s_inv = pow(target_s, -1, ORDER)
    
    # u1, u2 berechnen  
    u1 = (message_hash * s_inv) % ORDER
    u2 = (target_r * s_inv) % ORDER
    
    # Neuen Generator berechnen der gewuenschtes Ergebnis liefert
    # Dies ist eine vereinfachte Darstellung der komplexen Mathematik
    
    manipulated_gx = (target_r - u2 * PUBLIC_KEY_X) * pow(u1, -1, ORDER) % ORDER
    manipulated_gy = calculate_curve_point_y(manipulated_gx)
    
    return manipulated_gx, manipulated_gy

def calculate_curve_point_y(x, a=CURVE_A, b=CURVE_B, p=CURVE_P):
    """Berechnet y-Koordinate fuer gegebenen x-Wert auf elliptischer Kurve"""
    y_squared = (pow(x, 3, p) + a * x + b) % p
    y = pow(y_squared, (p + 1) // 4, p)  # Vereinfacht fuer p = 3 (mod 4)
    return y
\end{lstlisting}

\subsubsection{ASN.1-Struktur-Manipulation}

\begin{lstlisting}[language=python, caption=Zertifikat-Parameter überschreiben]
from cryptography import x509
from cryptography.hazmat.primitives import serialization

def create_manipulated_certificate(original_cert_path, manipulated_params):
    """
    Erstellt Zertifikat mit manipulierten ECC-Parametern
    """
    # Originales Zertifikat laden
    with open(original_cert_path, 'rb') as f:
        original_cert = x509.load_pem_x509_certificate(f.read())
    
    # ASN.1-Struktur manipulieren (vereinfacht)
    # In realer Implementierung wuerde hier die komplette
    # ASN.1-Struktur der ECC-Parameter ueberschrieben
    
    print("Manipulierte ECC-Parameter in Zertifikat:")
    print(f"Generator X: {manipulated_params['generator_x']}")
    print(f"Generator Y: {manipulated_params['generator_y']}")
    print(f"Curve A: {manipulated_params['a']}")
    print(f"Order: {manipulated_params['order']}")
    
    return "manipulated_certificate.pem"
\end{lstlisting}

\subsection{Sicherheitsimplikationen}

\subsubsection{Real-World Attack Scenarios}

\textbf{1. TLS-Certificate Spoofing:}
\begin{itemize}
    \item Angreifer erstellt gefälschtes TLS-Zertifikat für banking.example.com
    \item Manipuliert ECC-Parameter so, dass falsche Signatur gültig wird
    \item Windows-Clients akzeptieren das Zertifikat als vertrauenswürdig
    \item Man-in-the-Middle-Angriff erfolgreich
\end{itemize}

\textbf{2. Code-Signing Bypass:}
\begin{itemize}
    \item Malware wird mit manipuliertem Code-Signing-Zertifikat signiert  
    \item ECC-Parameter werden so gewählt, dass Windows die Signatur akzeptiert
    \item Malware wird als vertrauenswürdig eingestuft und ausgeführt
    \item Umgehung von Application Whitelisting
\end{itemize}

\textbf{3. E-Mail Signature Forgery:}
\begin{itemize}
    \item Gefälschte S/MIME-Zertifikate für E-Mail-Signierung
    \item Manipulation ermöglicht "gültige" Signaturen ohne Private Key
    \item Phishing-E-Mails erscheinen als von vertrauenswürdigen Absendern
\end{itemize}

\subsubsection{Betroffene Systeme}

\begin{itemize}
    \item \textbf{Windows 10:} Alle Versionen vor KB4534314 (Januar 2020)
    \item \textbf{Windows Server:} 2016, 2019, 2008 R2 SP1
    \item \textbf{Anwendungen:} Alle Software die Windows CryptoAPI verwendet
    \item \textbf{Browser:} Internet Explorer, Microsoft Edge (legacy)
    \item \textbf{E-Mail-Clients:} Outlook, Windows Mail
    \item \textbf{Development Tools:} Visual Studio, PowerShell
\end{itemize}

\subsection{Schutzmaßnahmen}

\subsubsection{Microsoft's Fix}

\textbf{Validierungsverbesserungen:}
\begin{itemize}
    \item \textbf{Parameter-Validierung:} Vollständige Prüfung aller ECC-Parameter gegen bekannte Standards
    \item \textbf{Generator-Verifikation:} Generator-Punkt wird gegen vertrauenswürdige CA-Parameter validiert
    \item \textbf{Kurven-Validierung:} Zusätzliche Checks für Kurvenkoeffizienten und mathematische Eigenschaften
    \item \textbf{Order-Verifikation:} Gruppenordnung wird gegen Standardwerte geprüft
\end{itemize}

\subsubsection{Best Practices für Entwickler}

\begin{lstlisting}[language=python, caption=Sichere Parameter-Validierung]
def validate_ecc_parameters(params, trusted_params):
    """
    Sichere Validierung von ECC-Parametern
    """
    # 1. Generator-Punkt validieren
    if (params['generator_x'] != trusted_params['generator_x'] or 
        params['generator_y'] != trusted_params['generator_y']):
        raise ValueError("Generator-Punkt entspricht nicht trusted Standard")
    
    # 2. Kurvenparameter check  
    if (params['a'] != trusted_params['a'] or 
        params['b'] != trusted_params['b']):
        raise ValueError("Kurvenparameter entsprechen nicht Standard")
    
    # 3. Ordnung validieren
    if params['order'] != trusted_params['order']:
        raise ValueError("Gruppenordnung entspricht nicht Standard")
    
    # 4. Primzahl checken
    if params['p'] != trusted_params['p']:
        raise ValueError("Grundkoerper entspricht nicht Standard")
    
    return True

# Verwendung in Signatur-Validierung
def secure_signature_validation(signature, message, public_key, cert_params):
    """Sichere ECDSA-Validierung mit Parameter-check"""

    # KRITISCH: Verwende trusted Parameter, nicht Zertifikat-Parameter
    trusted_params = get_trusted_curve_parameters('secp256r1')
    
    # Parameter validieren
    validate_ecc_parameters(cert_params, trusted_params)
    
    # Validierung mit trusted Parametern
    validator = ECCSignatureValidator(trusted_params)  # NICHT cert_params!
    return validator.validate_signature(message, signature.r, signature.s, 
                                      public_key.x, public_key.y)
\end{lstlisting}

\end{document}
