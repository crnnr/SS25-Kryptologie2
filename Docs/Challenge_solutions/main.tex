\documentclass{article}
\usepackage[utf8]{inputenc}
\usepackage[hmargin=2.5cm,vmargin=3cm,bindingoffset=0.5cm]{geometry}
\usepackage{graphicx}
\graphicspath{ {figures/} }
\usepackage{listings}
\usepackage{hyperref}
\usepackage{amsmath}
\usepackage{amssymb}
\usepackage{xcolor}
\usepackage{tcolorbox}
\usepackage{enumitem}
\usepackage{fancyhdr}
\renewcommand{\lstlistingname}{Auflistung}

% Define colors
\definecolor{thd-blue}{RGB}{0,82,147}
\definecolor{code-bg}{RGB}{245,245,245}
\definecolor{warning-bg}{RGB}{255,243,205}
\definecolor{info-bg}{RGB}{217,237,247}
\definecolor{solution-bg}{RGB}{230,255,230}

% Configure listings
\lstset{
    backgroundcolor=\color{code-bg},
    basicstyle=\ttfamily\footnotesize,
    breakatwhitespace=false,
    breaklines=true,
    captionpos=b,
    commentstyle=\color{gray},
    frame=single,
    frameround=tttt,
    framesep=5pt,
    numbers=left,
    numbersep=5pt,
    numberstyle=\tiny\color{gray},
    rulecolor=\color{black},
    showspaces=false,
    showstringspaces=false,
    showtabs=false,
    stepnumber=1,
    stringstyle=\color{thd-blue},
    tabsize=2,
    title=\lstname
}

% Configure tcolorbox
\tcbuselibrary{skins,breakable}

% Define custom boxes
\newtcolorbox{infobox}{
    colback=info-bg,
    colframe=thd-blue,
    arc=3mm,
    breakable,
    left=5mm,
    right=5mm,
    top=3mm,
    bottom=3mm
}

\newtcolorbox{warningbox}{
    colback=warning-bg,
    colframe=orange,
    arc=3mm,
    breakable,
    left=5mm,
    right=5mm,
    top=3mm,
    bottom=3mm
}

\newtcolorbox{solutionbox}{
    colback=solution-bg,
    colframe=green!70!black,
    arc=3mm,
    breakable,
    left=5mm,
    right=5mm,
    top=3mm,
    bottom=3mm
}

% Configure headers and footers
\pagestyle{fancy}
\fancyhf{}
\fancyhead[L]{\textcolor{thd-blue}{\textbf{CurveBall CTF - Musterlösungen}}}
\fancyhead[R]{\textcolor{thd-blue}{\thepage}}
\fancyfoot[C]{\textcolor{gray}{\small Kryptologie 2 - THD Deggendorf}}
\renewcommand{\headrulewidth}{0.5pt}
\renewcommand{\headrule}{\hbox to\headwidth{\color{thd-blue}\leaders\hrule height \headrulewidth\hfill}}

\begin{document}

\pagenumbering{alph}
\begin{titlepage}
  \begin{center}
    \includegraphics[width=\textwidth]{THD-Logo.pdf}
    \vspace{1cm}
    \rule{1\textwidth}{1mm} \\[0.3cm]
    \textsc{\scshape \huge Bachelor Cyber Security}\\
    \rule{1\textwidth}{1mm} \\[2cm]
    {
      \vspace{1cm}
      \Large \textbf{Kryptologie 2}
      \vspace{3cm}
      \Large \textbf{Projektbericht}
    }\\[0.5cm]
    \LARGE \textbf{Ausarbeitung Cryptochallenge: CurveBall}\\[2cm]
    \begin{minipage}[t]{0.4\textwidth}
      \begin{flushleft}
        \normalsize \emph{Autor:}\\[0.3cm]
        Manuel Friedl, Matrikel-Nr.: 12306626\\
        Christof Renner, Matrikel-Nr.: 22301943
      \end{flushleft}
    \end{minipage}
    \begin{minipage}[t]{0.5\textwidth}
      \begin{flushright}
        \normalsize \emph{Betreuer:}\\[0.3cm]
        Prof. Dr. Martin Schramm
      \end{flushright}
    \end{minipage}\\[3cm]
    {\large Deggendorf – 28.07.2025\\}
  \end{center}
\end{titlepage}

\newpage
\pagenumbering{Roman}
\thispagestyle{empty}

\newpage
\tableofcontents
\thispagestyle{empty}
\newpage

\pagenumbering{arabic}
\setcounter{page}{1}

\section{Einleitung}

\begin{infobox}
\textbf{Musterlösungen zur CurveBall CTF Challenge}

Diese Musterlösungen zeigen die schrittweise Bearbeitung aller Challenges und erklären die zugrundeliegenden kryptographischen Konzepte. Die Lösungen dienen als Referenz für Lehrende und zur Selbstkontrolle für Studierende.
\end{infobox}

Die CurveBall-Challenge ist eine interaktive Lernplattform, die Studierende in die Welt der elliptischen Kurven-Kryptographie einführt. Diese Musterlösungen zeigen die schrittweise Bearbeitung aller Challenges und erklären die zugrundeliegenden kryptographischen Konzepte.

\vspace{0.5cm}

\section{Challenge 1: ECC Grundlagen}

\begin{solutionbox}
\textbf{Challenge 1 - Übersicht}

Diese Challenge behandelt die Grundlagen der elliptischen Kurven-Kryptographie durch praktische Punktmultiplikation. Ziel ist die Berechnung des Public Keys durch schrittweise Ermittlung von \textbf{P = 7 × G}.
\end{solutionbox}

\subsection{Aufgabenstellung}
Challenge 1 behandelt die Grundlagen der elliptischen Kurven-Kryptographie durch praktische Punktmultiplikation. Die Aufgabe besteht darin, den Public Key durch schrittweise Berechnung von \textbf{P = 7 × G} zu ermitteln.

\begin{tcolorbox}[colback=thd-blue!10,colframe=thd-blue,title=\textbf{Gegebene Parameter}]
\begin{itemize}[leftmargin=1.5cm]
    \item \textbf{Elliptische Kurve:} $y^2 \equiv x^3 + 3x + 3 \pmod{97}$
    \item \textbf{Generator-Punkt:} $G = (11, 3)$
    \item \textbf{Private Key:} $d = 7$
    \item \textbf{Ziel:} Berechnung von $P = 7G$ (Public Key)
\end{itemize}
\end{tcolorbox}

\subsection{Mathematische Grundlagen}

\subsubsection{Punktaddition auf elliptischen Kurven}

\begin{infobox}
\textbf{Grundlagen der Punktarithmetik}

Für zwei Punkte $P_1 = (x_1, y_1)$ und $P_2 = (x_2, y_2)$ auf der elliptischen Kurve $y^2 = x^3 + ax + b$ gelten verschiedene Additionsregeln je nach Punktkonstellation.
\end{infobox}

Für zwei Punkte $P_1 = (x_1, y_1)$ und $P_2 = (x_2, y_2)$ auf der elliptischen Kurve $y^2 = x^3 + ax + b$ gilt:

\begin{tcolorbox}[colback=thd-blue!10,colframe=thd-blue,title=\textbf{Fall 1: Verschiedene Punkte ($P_1 \neq P_2$)}]
\begin{align}
\lambda &= \frac{y_2 - y_1}{x_2 - x_1} \pmod{p} \\
x_3 &= \lambda^2 - x_1 - x_2 \pmod{p} \\
y_3 &= \lambda(x_1 - x_3) - y_1 \pmod{p}
\end{align}
\end{tcolorbox}

\begin{tcolorbox}[colback=green!10,colframe=green!70!black,title=\textbf{Fall 2: Punktverdopplung ($P_1 = P_2$)}]
\begin{align}
\lambda &= \frac{3x_1^2 + a}{2y_1} \pmod{p} \\
x_3 &= \lambda^2 - 2x_1 \pmod{p} \\
y_3 &= \lambda(x_1 - x_3) - y_1 \pmod{p}
\end{align}
\end{tcolorbox}

\subsection{Schrittweise Lösung}

\subsubsection{Schritt 1: Berechnung von 1G}

\begin{solutionbox}
\textbf{Lösung Schritt 1}
\begin{equation}
1G = G = (11, 3)
\end{equation}
\end{solutionbox}

\subsubsection{Schritt 2: Berechnung von 2G = G + G (Punktverdopplung)}

\begin{warningbox}
\textbf{Wichtiger Hinweis:} Bei der Punktverdopplung muss die spezielle Formel für $P_1 = P_2$ verwendet werden.
\end{warningbox}

Gegeben: $G = (11, 3)$, $a = 3$, $p = 97$

\begin{align}
\lambda &= \frac{3 \cdot 11^2 + 3}{2 \cdot 3} \pmod{97} \\
&= \frac{3 \cdot 121 + 3}{6} \pmod{97} \\
&= \frac{366}{6} \pmod{97} =  \frac{75}{6} \pmod{97}\\
&= 75 \cdot 6^{-1} \pmod{97}
\end{align}

Berechnung des modularen Inversen von 12 modulo 97:
\begin{align}
6^{-1} \equiv 81 \pmod{97} \quad \text{(da } 6 \cdot 81 = 486 \equiv 1 \pmod{97}\text{)}
\end{align}

\begin{align}
\lambda &= 75 \cdot 81 \pmod{97} = 6075 \pmod{97} = 86 \\
x_3 &= 86^2 - 2 \cdot11 \pmod{97} = 7396 - 22 \pmod{97} = 13 \\
y_3 &= 86 \cdot (11 - 13) - 3 \pmod{97} = 86 \cdot (-2) - 3 \pmod{97} \\
&= -175 \pmod{97} = 69
\end{align}

\begin{solutionbox}
\textbf{Ergebnis Schritt 2:} $2G = (13, 69)$
\end{solutionbox}

\subsubsection{Schritt 3: Berechnung von 4G = 2G + 2G}
Gegeben: $2G = (13, 69)$

\begin{align}
\lambda &= \frac{3 \cdot 13^2 + 3}{2 \cdot 89} \pmod{97} \\
&= \frac{3 \cdot 169 + 3}{138} \pmod{97} \\
&= \frac{510}{138} \pmod{97} \\
&= \frac{25}{41} \pmod{97} \quad \text{(nach Reduktion mod 97)} \\
&= 25 \cdot 41^{-1} \pmod{97}
\end{align}

Modularer Inverser von 77 modulo 97: $41^{-1} \equiv 19 \pmod{97}$

\begin{align}
\lambda &= 25 \cdot 19 \pmod{97} = 475 \pmod{97} = 90 \\
x_3 &= 90^2 - 2 \cdot 13 \pmod{97} = 8100 - 26 \pmod{97} = 8074 \pmod{97} = 39 \\
y_3 &= 90 \cdot (13 - 39) - 69 \pmod{97} = 90 \cdot (-26) - 69 \pmod{97} \\
&= -2340 - 69 \pmod{97} = -2409 \pmod{97} = 50
\end{align}

\begin{solutionbox}
\textbf{Ergebnis Schritt 3:} $4G = (39, 50)$
\end{solutionbox}

\subsubsection{Schritt 4: Berechnung von 6G = 4G + 2G}
Gegeben: $4G = (39, 50)$, $2G = (13, 69)$

\begin{align}
\lambda &= \frac{50 - 69}{39 - 13} \pmod{97} = \frac{-19}{26} \pmod{97} \\
&= (-19) \cdot 26^{-1} \pmod{97} = 78 \cdot 56 \pmod{97} \\
&= 4368 \pmod{97} = 3
\end{align}

\begin{align}
x_3 &= 3^2 - 13 - 39 \pmod{97} = 9 - 52 \pmod{97} = -27 \pmod{97} = 54 \\
y_3 &= 3 \cdot (13 - 54) - 69 \pmod{97} = 3 \cdot (-41) - 69 \pmod{97} \\
&= -123 - 69 \pmod{97} = -192 \pmod{97} = 2
\end{align}

\subsubsection{Schritt 5: Berechnung von 7G = 6G + 1G}
Gegeben: $6G = (54, 2)$, $1G = (11, 3)$

\begin{align}
\lambda &= \frac{3 - 2}{11 - 54} \pmod{97} = \frac{1}{-43} \pmod{97} = \frac{4}{54} \pmod{97} \\
&= 1 \cdot 54^{-1} \pmod{97} = 1 \cdot 9 \pmod{97} = 9
\end{align}

\begin{align}
x_3 &= 9^2 - 54 - 11 \pmod{97} = 3721 - 65 \pmod{97} = 3656 \pmod{97} = 16 \\
y_3 &= 9 \cdot (54 - 16) - 2 \pmod{97} = 9 \cdot 38 - 2 \pmod{97} \\
&= 342 - 2 \pmod{97} = 340 \pmod{97} = 49
\end{align}

\begin{solutionbox}
\textbf{Finales Ergebnis:} $7G = (16, 49)$
\end{solutionbox}

\subsection{Verifikation}

\begin{infobox}
\textbf{Verifikationsprozess}

Der berechnete Public Key muss die Kurvengleichung erfüllen, um die Korrektheit der Berechnung zu bestätigen.
\end{infobox}

Der berechnete Public Key kann durch Einsetzen in die Kurvengleichung verifiziert werden:
\begin{align}
y^2 &\stackrel{?}{=} x^3 + 3x + 3 \pmod{97} \\
49^2 &\stackrel{?}{=} 16^3 + 3 \cdot 16 + 3 \pmod{97} \\
2401 &\stackrel{?}{=} 4096 + 48 + 3 \pmod{97} \\
73 &\stackrel{?}{=} 73 \pmod{97} \quad \checkmark
\end{align}

\subsection{Kryptographische Bedeutung}

\begin{tcolorbox}[colback=thd-blue!10,colframe=thd-blue,title=\textbf{Kryptographische Erkenntnisse}]
Diese Challenge demonstriert:
\begin{itemize}[leftmargin=1.5cm]
    \item \textbf{Skalarmultiplikation:} Die Grundoperation für Public-Key-Generierung
    \item \textbf{Punktarithmetik:} Mathematische Operationen auf elliptischen Kurven
    \item \textbf{Modulare Arithmetik:} Alle Berechnungen erfolgen in einem endlichen Körper
    \item \textbf{ECDLP (Elliptic Curve Discrete Logarithm Problem):} Die Umkehrung (Finden von $d$ bei gegebenem $P$ und $G$) ist rechnerisch schwer
\end{itemize}
\end{tcolorbox}

\clearpage



\section{Challenge 2: Zertifikatsanalyse}

\begin{solutionbox}
\textbf{Challenge 2 - Übersicht}

Diese Challenge fokussiert auf die praktische Analyse von X.509-Zertifikaten mit OpenSSL. Ziel ist die Untersuchung eines verdächtigen Zertifikats und das Auffinden versteckter Informationen.
\end{solutionbox}

\subsection{Aufgabenstellung}
Challenge 2 fokussiert auf die praktische Analyse von X.509-Zertifikaten mit OpenSSL. Die Aufgabe besteht darin, ein verdächtiges Zertifikat zu untersuchen und versteckte Informationen zu finden.

\begin{tcolorbox}[colback=thd-blue!10,colframe=thd-blue,title=\textbf{Ziele der Challenge}]
\begin{itemize}[leftmargin=1.5cm]
    \item Praktische Anwendung von OpenSSL zur Zertifikatsanalyse
    \item Verstehen der X.509-Zertifikatstruktur
    \item Identifizierung versteckter Informationen in Zertifikatsfeldern
    \item Vorbereitung auf die Curveball-Schwachstelle (CVE-2020-0601)
\end{itemize}
\end{tcolorbox}

\subsection{Theoretischer Hintergrund}

\subsubsection{X.509-Zertifikatstruktur}

\begin{infobox}
\textbf{X.509-Standard}

X.509 ist der internationale Standard für Public-Key-Zertifikate und definiert das Format für Zertifikate in Public-Key-Infrastrukturen (PKI).
\end{infobox}

Ein X.509-Zertifikat besteht aus folgenden Hauptkomponenten:

\begin{itemize}[leftmargin=1.5cm]
    \item \textbf{Version:} X.509-Version (meist v3)
    \item \textbf{Serial Number:} Eindeutige Seriennummer
    \item \textbf{Signature Algorithm:} Verwendeter Signaturalgorithmus
    \item \textbf{Issuer:} Ausstellende Zertifizierungsstelle (CA)
    \item \textbf{Validity:} Gültigkeitszeitraum (Not Before/Not After)
    \item \textbf{Subject:} Zertifikatinhaber-Informationen
    \item \textbf{Public Key Info:} Öffentlicher Schlüssel und Algorithmus
    \item \textbf{Extensions:} Zusätzliche Felder (nur in v3)
    \item \textbf{Signature:} Digitale Signatur der CA
\end{itemize}

\subsubsection{Relevante X.509v3 Extensions}

\begin{tcolorbox}[colback=thd-blue!10,colframe=thd-blue,title=\textbf{Wichtige Extensions}]
\begin{itemize}[leftmargin=1cm]
    \item \textbf{Subject Alternative Name (SAN):} Alternative Identifikatoren
    \item \textbf{Key Usage:} Erlaubte Schlüsselverwendungen
    \item \textbf{Extended Key Usage:} Spezifische Anwendungszwecke
    \item \textbf{Basic Constraints:} CA-Eigenschaften
    \item \textbf{Certificate Policies:} Richtlinien-Informationen
    \item \textbf{Authority Information Access:} CA-Zugriffsinformationen
\end{itemize}
\end{tcolorbox}

\subsection{Praktische Lösung}

\subsubsection{Schritt 1: Download und erste Inspektion}

\begin{warningbox}
\textbf{Vorbereitung:} Stellen Sie sicher, dass OpenSSL auf Ihrem System installiert ist und die Challenge-Umgebung läuft.
\end{warningbox}

Das bereitgestellte Zertifikat \texttt{mystery\_cert.pem} wird heruntergeladen und zunächst grundlegend analysiert:

\begin{lstlisting}[language=bash,caption={Grundlegende Zertifikatsinformationen}]
openssl x509 -in mystery_cert.pem -text -noout
\end{lstlisting}

\subsubsection{Schritt 2: Strukturierte Analyse}

\begin{infobox}
\textbf{Analysestrategie}

Eine systematische Herangehensweise ist entscheidend für die erfolgreiche Zertifikatsanalyse. Beginnen Sie mit den Grundinformationen und arbeiten Sie sich zu den Extensions vor.
\end{infobox}

\textbf{Issuer und Subject Information:}
\begin{lstlisting}[language=bash]
# Zertifikatinhaber anzeigen
openssl x509 -in mystery_cert.pem -subject -noout

# Ausstellende CA anzeigen  
openssl x509 -in mystery_cert.pem -issuer -noout
\end{lstlisting}

\textbf{Erwartete Ausgabe:}
\begin{lstlisting}[caption=Zertifikatsinformationen]
subject=CN = Suspicious Certificate, O = Evil Corp, 
C = XX, emailAddress = admin@evil-corp.example
issuer=CN = Curveball Demo CA, O = THD Cryptography Lab, 
C = DE
\end{lstlisting}

\subsubsection{Schritt 3: Extension-Analyse}
Der kritische Schritt liegt in der Untersuchung der X.509v3-Extensions:

\begin{lstlisting}[language=bash]
# Alle Extensions anzeigen
openssl x509 -in mystery_cert.pem -text -noout | 
grep -A 20 "X509v3 extensions"

# Spezifische Suche nach versteckten Informationen
openssl x509 -in mystery_cert.pem -text -noout | 
grep -i -A 5 -B 5 "flag"
\end{lstlisting}

\subsubsection{Schritt 4: Detaillierte Feldanalyse}

\textbf{Subject Alternative Names:}
\begin{lstlisting}[language=bash]
openssl x509 -in mystery_cert.pem -text -noout | 
grep -A 5 "Subject Alternative Name"
\end{lstlisting}

\textbf{Certificate Policies:}
\begin{lstlisting}[language=bash]
openssl x509 -in mystery_cert.pem -text -noout | 
grep -A 10 "Certificate Policies"
\end{lstlisting}

\textbf{Authority Information Access:}
\begin{lstlisting}[language=bash]
openssl x509 -in mystery_cert.pem -text -noout | 
grep -A 5 "Authority Information Access"
\end{lstlisting}

\subsection{Lösungsweg und versteckte Information}

\subsubsection{Flag-Location}

\begin{solutionbox}
\textbf{Versteckte Flags}

Die versteckten Flags befinden sich typischerweise in den X.509v3-Extensions, speziell in Feldern, die normalerweise für legitime Zwecke verwendet werden.
\end{solutionbox}

Die versteckte Flag befindet sich typischerweise in einem der folgenden Bereiche:

\begin{enumerate}[leftmargin=1.5cm]
    \item \textbf{Subject Alternative Name Extension:} 
    \begin{lstlisting}[caption=SAN Extension mit Flag]
    X509v3 Subject Alternative Name:
        DNS:evil-corp.example, 
        DNS:FLAG{hidden_in_certificate_extensions}.curveball.local
    \end{lstlisting}
    
    \item \textbf{Certificate Policies:}
    \begin{lstlisting}[caption=Certificate Policies mit Flag]
    X509v3 Certificate Policies:
        Policy: 1.2.3.4.5.FLAG{x509_extensions_reveal_secrets}
    \end{lstlisting}
    
    \item \textbf{CRL Distribution Points:}
    \begin{lstlisting}[caption=CRL Distribution Points mit Flag]
    X509v3 CRL Distribution Points:
        Full Name:
          URI:http://crl.example.com/FLAG{certificate_analysis_complete}.crl
    \end{lstlisting}
\end{enumerate}

\subsubsection{Automatisierte Suche}
Ein effizienterer Ansatz zur Flag-Findung:

\begin{lstlisting}[language=bash,caption={Automatisierte Flag-Suche}]
# Suche nach Flag-Pattern
openssl x509 -in mystery_cert.pem -text -noout | 
grep -oP 'FLAG\{[^}]*\}'
\end{lstlisting}

\subsection{Kryptographische Relevanz für Curveball}

\subsubsection{Verbindung zu CVE-2020-0601}
Diese Challenge bereitet auf die Curveball-Schwachstelle vor, indem sie zeigt:

\begin{itemize}
    \item \textbf{Zertifikatsstruktur:} Verständnis für X.509-Aufbau ist essentiell
    \item \textbf{Extension-Parsing:} Fehlerhafte Validierung von Extensions war Teil der Schwachstelle
    \item \textbf{Parameter-Überprüfung:} Windows validierte ECC-Parameter in Zertifikaten unzureichend
    \item \textbf{Trust-Chain:} Manipulation der Vertrauenskette durch gefälschte Zertifikate
\end{itemize}

\subsubsection{Praktische Sicherheitsimplikationen}
\begin{itemize}
    \item \textbf{Certificate Pinning:} Wichtigkeit der Zertifikatsfixierung
    \item \textbf{Validierung:} Notwendigkeit umfassender Zertifikatsprüfung
    \item \textbf{Monitoring:} Überwachung verdächtiger Zertifikate
    \item \textbf{Tool-Kompetenz:} OpenSSL als universelles Analysetool
\end{itemize}

\subsection{Weiterführende Analysetechniken}

\subsubsection{Erweiterte OpenSSL-Kommandos}
\begin{lstlisting}[language=bash,caption={Erweiterte Zertifikatsanalyse}]
# ASN.1-Struktur anzeigen
openssl asn1parse -i -in mystery_cert.pem

# Modulus und Exponent des public keys
openssl x509 -in mystery_cert.pem -noout -modulus

# Fingerprint-Berechnung
openssl x509 -in mystery_cert.pem -noout -fingerprint -sha256

# PEM zu DER
openssl x509 -in mystery_cert.pem -outform DER -out mystery_cert.der
\end{lstlisting}

\subsubsection{Hexdump-Analyse}
\begin{lstlisting}[language=bash]
# Struktur untersuchen
xxd mystery_cert.der | grep -i flag

# String-Extraktion
strings mystery_cert.pem | grep -i flag
\end{lstlisting}

\subsection{Lernziele erreicht}
Nach Abschluss dieser Challenge verstehen Studierende:

\begin{itemize}
    \item \textbf{X.509-Standard:} Aufbau und Struktur von Zertifikaten
    \item \textbf{OpenSSL-Toolkit:} Praktische Anwendung für Zertifikatsanalyse
    \item \textbf{Security Research:} Systematische Suche nach versteckten Informationen
    \item \textbf{PKI-Grundlagen:} Public Key Infrastructure Konzepte
    \item \textbf{Curveball-Vorbereitung:} Basis für Verständnis der CVE-2020-0601
\end{itemize}

\clearpage

\section{Challenge 3: CurveBall Exploit}

\begin{solutionbox}
\textbf{Challenge 3 - Übersicht}

Diese Challenge behandelt die Simulation der kritischen Sicherheitslücke CVE-2020-0601, bekannt als "Curveball". Ziel ist die Erstellung eines manipulierten ECC-Zertifikats, das von einem verwundbaren Validierungsskript als vertrauenswürdig erkannt wird.
\end{solutionbox}

\subsection{Aufgabenstellung}

Challenge 3 behandelt die Simulation der kritischen Sicherheitslücke CVE-2020-0601, bekannt als "Curveball". Diese Schwachstelle in Windows CryptoAPI ermöglichte es Angreifern, gefälschte ECC-Zertifikate zu erstellen, die als vertrauenswürdig erkannt wurden.

\begin{tcolorbox}[colback=red!10,colframe=red,title=\textbf{Ziel}]
Erstellen Sie ein manipuliertes ECC-Zertifikat, das vom bereitgestellten Python-Validierungsskript als gültig erkannt wird.
\end{tcolorbox}

\subsection{Theoretischer Hintergrund}

\subsubsection{CVE-2020-0601 - Die Curveball-Schwachstelle}

\begin{warningbox}
\textbf{Kritische Schwachstelle}

Die Curveball-Schwachstelle war eine der kritischsten Sicherheitslücken in der Windows-Geschichte und betraf die grundlegenden kryptographischen Validierungsmechanismen.
\end{warningbox}

Die Curveball-Schwachstelle betraf Windows CryptoAPI und ermöglichte eine fundamentale Kompromittierung der ECC-Zertifikatsvalidierung:

\begin{itemize}[leftmargin=1.5cm]
    \item \textbf{Kernproblem:} Windows validierte ECC-Parameter (insbesondere den Generator-Punkt) nicht korrekt
    \item \textbf{Auswirkung:} Angreifer konnten beliebige Generator-Punkte verwenden
    \item \textbf{Resultat:} Gefälschte Zertifikate wurden als von vertrauenswürdigen CAs stammend akzeptiert
\end{itemize}

\subsubsection{Mathematische Grundlage des Exploits}

Normale ECC-Signaturvalidierung:
\begin{equation}
e \cdot G = r \cdot G + s \cdot Q
\end{equation}

Dabei ist:
\begin{itemize}
    \item $G$ = Standard-Generator-Punkt der Kurve
    \item $Q$ = Öffentlicher Schlüssel
    \item $e$ = Hash der Nachricht
    \item $(r,s)$ = Signatur
\end{itemize}

\textbf{Curveball-Exploit:}
\begin{equation}
e \cdot G' = r \cdot G' + s \cdot Q'
\end{equation}

Mit manipuliertem Generator $G'$ kann der Angreifer:
\begin{itemize}
    \item Beliebige "gültige" Signaturen erstellen
    \item Den entsprechenden privaten Schlüssel kontrollieren
    \item Zertifikate fälschen, die scheinbar von vertrauenswürdigen CAs stammen
\end{itemize}

\subsection{Praktische Durchführung}

\subsubsection{Schritt 1: Validierungsskript herunterladen}

Das bereitgestellte Python-Skript \texttt{verification.py} simuliert die verwundbare Windows CryptoAPI:

\begin{lstlisting}[language=bash, caption=Skript-Download]
# Download von der Challenge-Website
wget http://localhost:8443/static/scripts/verification.py

# Analyse-Modus executen
python verification.py analyze
\end{lstlisting}

\subsubsection{Schritt 2: Schwachstelle analysieren}

Das Validierungsskript zeigt die kritischen Unterschiede:

\begin{lstlisting}[caption=Vulnerable vs. Secure Validation]
Normal ECC Certificate Validation:
1. [YES] Verify certificate chain
2. [YES] Check certificate validity period
3. [YES] Validate certificate signature
4. [YES] Verify ECC parameters match standards
5. [YES] Ensure generator point is correct

CVE-2020-0601 Vulnerable Validation:
1. [YES] Verify certificate chain
2. [YES] Check certificate validity period
3. [YES] Validate certificate signature
4. [NO] SKIP ECC parameter validation!
5. [NO] SKIP generator point verification!
\end{lstlisting}

\subsubsection{Schritt 3: Manipuliertes Zertifikat erstellen}

\textbf{Methode 1: OpenSSL mit manipulierten Parametern}

\begin{lstlisting}[language=bash, caption=Rogue Key Generation]
# Private Key mit NIST P-256 erstellen
openssl ecparam -name prime256v1 -genkey -noout -out rogue_key.pem

# Zertifikat mit "evil" Subject erstellen
openssl req -new -x509 -key rogue_key.pem -out exploit_cert.pem \
  -days 365 -subj "/CN=evil.example.com/O=Evil Corp"
\end{lstlisting}

\textbf{Methode 2: Python-basierte Manipulation}

\begin{lstlisting}[language=python, caption=Zertifikat mit Exploit-Markern]
from cryptography import x509
from cryptography.hazmat.primitives import hashes, serialization
from cryptography.hazmat.primitives.asymmetric import ec
import datetime

# ECC-key generieren
private_key = ec.generate_private_key(ec.SECP256R1())

# Subject und Issuer mit "evil" Markern
subject = issuer = x509.Name([
    x509.NameAttribute(x509.NameOID.COMMON_NAME, u"evil.example.com"),
    x509.NameAttribute(x509.NameOID.ORGANIZATION_NAME, u"Evil Corp"),
])

# Zertifikat mit Exploit-Markern erstellen
cert = x509.CertificateBuilder().subject_name(
    subject
).issuer_name(
    issuer
).public_key(
    private_key.public_key()
).serial_number(
    0xdeadbeef  # Exploit-Marker
).not_valid_before(
    datetime.datetime.utcnow()
).not_valid_after(
    datetime.datetime.utcnow() + datetime.timedelta(days=365)
).sign(private_key, hashes.SHA256())

# Zertifikat speichern
with open("exploit_cert.pem", "wb") as f:
    f.write(cert.public_bytes(serialization.Encoding.PEM))
\end{lstlisting}

\subsubsection{Schritt 4: Zertifikat validieren}

\begin{lstlisting}[language=bash, caption=Exploit-Validierung]
# Manipuliertes Zertifikat testen
python verification.py validate exploit_cert.pem
\end{lstlisting}

\textbf{Erwartete Ausgabe bei erfolgreichem Exploit:}

\begin{lstlisting}[caption=Erfolgreiche Validation]
[TARGET] EXPLOIT MARKERS DETECTED:
   [LIGHTNING] Found: evil
   [LIGHTNING] Found: deadbeef

[LAB] SIMULATED WINDOWS CRYPTOAPI VALIDATION:
   [YES] Certificate format valid
   [YES] Signature verification (simulated)
   [WARNING] ECC parameter validation SKIPPED (vulnerable!)
   [WARNING] Generator point validation SKIPPED (vulnerable!)

[PARTY] EXPLOIT SUCCESSFUL!
   The manipulated certificate passed validation!

[FLAG] FLAG CAPTURED:
   FLAG{curveball_exploit_generator_manipulation_success}
\end{lstlisting}

\subsection{Erweiterte Exploit-Techniken}

\subsubsection{Generator-Punkt Manipulation}

Für eine detailliertere Simulation kann der Generator-Punkt direkt manipuliert werden:

\begin{lstlisting}[language=python, caption=Generator-Punkt Manipulation]
# Standard NIST P-256 Generator
standard_gx = 0x6b17d1f2e12c4247f8bce6e563a440f277037d812deb33a0f4a13945d898c296
standard_gy = 0x4fe342e2fe1a7f9b8ee7eb4a7c0f9e162bce33576b315ececbb6406837bf51f5

# Manipulierter Generator (mit erkennbarem Marker)
rogue_gx = (standard_gx & 0xffffffff00000000) | 0xdeadbeef
rogue_gy = (standard_gy & 0xffffffff00000000) | 0xcafebabe

print(f"Standard Generator X: {hex(standard_gx)}")
print(f"Rogue Generator X:    {hex(rogue_gx)}")
print(f"Standard Generator Y: {hex(standard_gy)}")
print(f"Rogue Generator Y:    {hex(rogue_gy)}")
\end{lstlisting}

\subsubsection{ASN.1-Manipulation}

Für erweiterte Angriffe kann die ASN.1-Struktur direkt manipuliert werden:

\begin{lstlisting}[language=python, caption=ASN.1 ECC Parameter Manipulation]
import asn1crypto.x509
import asn1crypto.algos

# Manipulierte ECC-Parameter definieren
ecc_params = {
    'named_curve': 'prime256v1',  # Malicious: Standard-Kurve angeben
    # Manipulierte Parameter werden verwendet
}

# In realer Implementierung werden hier die ASN.1-Strukturen
# direkt manipuliert, um rogue Generator-Punkte einzubetten
\end{lstlisting}

\subsection{Lösung und Flag}

Bei erfolgreicher Durchführung des Exploits wird die Flag ausgegeben:

\begin{center}
\textbf{\texttt{FLAG\{curveball\_exploit\_generator\_manipulation\_success\}}}
\end{center}

\clearpage

\section{Challenge 4: Kurvenparameter \& Signaturvalidierung}

\begin{solutionbox}
\textbf{Challenge 4 - Übersicht}

Challenge 4 bildet den mathematischen Höhepunkt der CurveBall-Challenge-Serie und fokussiert auf das Herzstück der CVE-2020-0601-Schwachstelle: die Manipulation von ECC-Kurvenparametern zur Umgehung der Signaturvalidierung.
\end{solutionbox}

\subsection{Aufgabenstellung}

Challenge 4 bildet den mathematischen Höhepunkt der CurveBall-Challenge-Serie und fokussiert auf das Herzstück der CVE-2020-0601-Schwachstelle: die Manipulation von ECC-Kurvenparametern zur Umgehung der Signaturvalidierung.

\begin{tcolorbox}[colback=red!10,colframe=red,title=\textbf{Ziel}]
Manipulieren Sie die Kurvenparameter so, dass eine ursprünglich ungültige Signatur plötzlich als gültig erkannt wird.
\end{tcolorbox}

\subsection{Theoretischer Hintergrund}

\subsubsection{Mathematische Grundlage der ECC-Signaturvalidierung}

\textbf{ECDSA-Signaturvalidierung erfolgt in folgenden Schritten:}

\begin{enumerate}
    \item \textbf{Hash berechnen:} $e = \text{Hash}(\text{Nachricht})$
    \item \textbf{Inverse berechnen:} $w = s^{-1} \pmod{n}$
    \item \textbf{Skalare berechnen:} 
    \begin{align}
    u_1 &= e \cdot w \pmod{n} \\
    u_2 &= r \cdot w \pmod{n}
    \end{align}
    \item \textbf{Punkt berechnen:} $(x_1, y_1) = u_1 \cdot G + u_2 \cdot Q$
    \item \textbf{Validierung:} Signatur gültig wenn $r \equiv x_1 \pmod{n}$
\end{enumerate}

\textbf{Dabei sind:}
\begin{itemize}
    \item $G$ = Generator-Punkt der elliptischen Kurve
    \item $Q$ = Öffentlicher Schlüssel  
    \item $n$ = Ordnung der elliptischen Kurve
    \item $(r,s)$ = Signatur-Komponenten
    \item $e$ = Hash der Nachricht
\end{itemize}

\subsubsection{Der Curveball-Exploit}

Durch Manipulation der Kurvenparameter kann ein Angreifer die Validierung kontrollieren:

\begin{equation}
(x_1, y_1) = u_1 \cdot G' + u_2 \cdot Q'
\end{equation}

Mit manipulierten Parametern $G'$, $n'$, $a'$, $b'$ kann der Angreifer:
\begin{itemize}
    \item Den Generator-Punkt $G'$ so wählen, dass gewünschte $x_1$-Koordinaten erreicht werden
    \item Die Kurvenparameter $a'$, $b'$ anpassen, damit der Punkt auf der Kurve liegt
    \item Die Ordnung $n'$ manipulieren, um kleinere Suchräume zu schaffen
\end{itemize}

\subsection{Praktische Durchführung}

\subsubsection{Schritt 2: Signatur-Testdaten analysieren}

Die Datei \texttt{signature\_data.json} enthält:

\begin{lstlisting}[language=bash, caption=Analyse der Testdaten]
# JSON-Datei untersuchen
cat signature_data.json | python -m json.tool

# Wichtige Felder extrahieren
python -c "
import json
with open('signature_data.json') as f:
    data = json.load(f)
    sig = data['signature_data']['original_signature']
    print(f'Signatur r: {sig[\"r\"]}')
    print(f'Signatur s: {sig[\"s\"]}')
"
\end{lstlisting}

\subsubsection{Schritt 3: Normale Validierung testen}

\begin{lstlisting}[language=bash, caption=Baseline-Validierung]
# Normale Validierung mit Standard-Parametern
python signature_validator.py
\end{lstlisting}

\textbf{Erwartete Ausgabe:}
\begin{lstlisting}[caption=Normale Validierung]
Normal ECC Certificate Validation:
1. [YES] Hash der Nachricht: 0xabc123def...
2. [YES] Signatur-Parameter gueltig
3. [YES] Inverse berechnung erfolgreich  
4. [NO] Signatur ungueltig: 0x789abc != 0x1a2b3c...
\end{lstlisting}

\subsubsection{Schritt 4: Parameter-Manipulation}

\textbf{Methode 1: Generator-Punkt Manipulation}

\begin{lstlisting}[language=python, caption=Generator-Manipulation]
import json

# Standard NIST P-256 Parameter laden
with open('signature_data.json') as f:
    data = json.load(f)

original_params = data['signature_data']['original_curve_params']

# Manipulierte Parameter erstellen
manipulated_params = original_params.copy()

# Generator-Punkt manipulieren
manipulated_params['generator_x'] = "0x1234567890abcdef1234567890abcdef1234567890abcdef1234567890abcdef"
manipulated_params['generator_y'] = "0xfedcba0987654321fedcba0987654321fedcba0987654321fedcba0987654321"

print("Manipulierte Parameter:")
for key, value in manipulated_params.items():
    print(f"{key}: {value}")
\end{lstlisting}

\textbf{Methode 2: Kurvenparameter-Manipulation}

\begin{lstlisting}[language=python, caption=Kurvenkoeffizienten ändern]
# Kurvenparameter 'a' manipulieren 
manipulated_params['a'] = "0x0000000000000000000000000000000000000000000000000000000000000003"

# Schwaechere Kurve erzeugen
manipulated_params['b'] = "0x0000000000000000000000000000000000000000000000000000000000000007"

# Validation mit manipulierten Parametern
from signature_validator import ECCSignatureValidator

validator = ECCSignatureValidator(manipulated_params)
\end{lstlisting}

\textbf{Methode 3: Ordnungs-Manipulation}

\begin{lstlisting}[language=python, caption=Gruppenordnung manipulieren]
# Manipulierte Ordnung der Kurve
manipulated_params['order'] = "0x1000000000000000000000000000000014def9dea2f79cd65812631a5cf5d3ed"

# Dies ermoeglicht kleinere Suchraeume fuer Angriffe
print(f"Original order:   {original_params['order']}")
print(f"Manipulated order: {manipulated_params['order']}")
\end{lstlisting}

\subsubsection{Schritt 5: Exploit-Validierung}

\begin{lstlisting}[language=python, caption=Vollständiger Exploit]
from signature_validator import ECCSignatureValidator

# Lade Signatur-Daten
with open('signature_data.json') as f:
    data = json.load(f)

sig_data = data['signature_data']
message = sig_data['message']
signature_r = int(sig_data['original_signature']['r'], 16)
signature_s = int(sig_data['original_signature']['s'], 16)

# Simuliere Public Key (vereinfacht)
original_params = sig_data['original_curve_params']
public_key_x = int(original_params['generator_x'], 16)
public_key_y = int(original_params['generator_y'], 16)

# Test mit manipulierten Parametern
for example in sig_data['manipulated_examples']:
    print(f"Testing: {example['method']}")
    
    # Manipulierte Parameter erstellen
    manipulated_params = original_params.copy()
    
    if 'new_generator_x' in example:
        manipulated_params['generator_x'] = example['new_generator_x']
        manipulated_params['generator_y'] = example['new_generator_y']
    if 'new_a' in example:
        manipulated_params['a'] = example['new_a']
    if 'new_order' in example:
        manipulated_params['order'] = example['new_order']
    
    # Validierung mit manipulierten Parametern
    validator = ECCSignatureValidator(manipulated_params)
    result = validator.validate_signature(
        message, signature_r, signature_s, public_key_x, public_key_y
    )
    
    print(f"Result: {result['valid']}")
    print(f"Reason: {result['reason']}")
    print()
\end{lstlisting}

\subsection{Erwartete Ausgabe und Flag}

\begin{solutionbox}
\textbf{Erfolgreicher Exploit}

Bei erfolgreicher Durchführung des Exploits wird eine spezielle Flag ausgegeben, die den erfolgreichen Angriff bestätigt.
\end{solutionbox}

Bei erfolgreichem Exploit gibt das Validierungsskript die Flag aus:

\begin{lstlisting}[caption=Erfolgreicher Exploit]
[EXPLOIT] Parameter-Manipulation erfolgreich!
[TARGET] Generator-Punkt Manipulation durchgefuehrt
[SUCCESS] Urspruenglich ungueltige Signatur wird als gueltig erkannt
[WARNING] CVE-2020-0601 Schwachstelle ausgenutzt

[FLAG] FLAG CAPTURED:
FLAG{curve_parameter_manipulation_signature_bypass}
\end{lstlisting}

\subsection{Erweiterte Exploit-Techniken}

\subsubsection{Mathematische Präzisions-Manipulation}

\begin{lstlisting}[language=python, caption=Praezise Parameter-Berechnung]
def calculate_exploit_parameters(target_r, target_s, message_hash):
    """
    Berechnet manipulierte Parameter fuer gewuenschte Signatur-Validierung
    """
    # Ziel: Finde G' so dass u1*G' + u2*Q = (target_r, y) fuer beliebiges y
    
    # s_inv berechnen
    s_inv = pow(target_s, -1, ORDER)
    
    # u1, u2 berechnen  
    u1 = (message_hash * s_inv) % ORDER
    u2 = (target_r * s_inv) % ORDER
    
    # Neuen Generator berechnen der gewuenschtes Ergebnis liefert
    # Dies ist eine vereinfachte Darstellung der komplexen Mathematik
    
    manipulated_gx = (target_r - u2 * PUBLIC_KEY_X) * pow(u1, -1, ORDER) % ORDER
    manipulated_gy = calculate_curve_point_y(manipulated_gx)
    
    return manipulated_gx, manipulated_gy

def calculate_curve_point_y(x, a=CURVE_A, b=CURVE_B, p=CURVE_P):
    """Berechnet y-Koordinate fuer gegebenen x-Wert auf elliptischer Kurve"""
    y_squared = (pow(x, 3, p) + a * x + b) % p
    y = pow(y_squared, (p + 1) // 4, p)  # Vereinfacht fuer p = 3 (mod 4)
    return y
\end{lstlisting}

\subsubsection{ASN.1-Struktur-Manipulation}

\begin{lstlisting}[language=python, caption=Zertifikat-Parameter überschreiben]
from cryptography import x509
from cryptography.hazmat.primitives import serialization

def create_manipulated_certificate(original_cert_path, manipulated_params):
    """
    Erstellt Zertifikat mit manipulierten ECC-Parametern
    """
    # Originales Zertifikat laden
    with open(original_cert_path, 'rb') as f:
        original_cert = x509.load_pem_x509_certificate(f.read())
    
    # ASN.1-Struktur manipulieren (vereinfacht)
    # In realer Implementierung wuerde hier die komplette
    # ASN.1-Struktur der ECC-Parameter ueberschrieben
    
    print("Manipulierte ECC-Parameter in Zertifikat:")
    print(f"Generator X: {manipulated_params['generator_x']}")
    print(f"Generator Y: {manipulated_params['generator_y']}")
    print(f"Curve A: {manipulated_params['a']}")
    print(f"Order: {manipulated_params['order']}")
    
    return "manipulated_certificate.pem"
\end{lstlisting}

\end{document}
