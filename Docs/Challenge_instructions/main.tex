\documentclass{article}
\usepackage[utf8]{inputenc}
\usepackage[hmargin=2.5cm,vmargin=3cm,bindingoffset=0.5cm]{geometry}
\usepackage{graphicx}
\graphicspath{ {figures/} }
\usepackage{listings}
\usepackage{hyperref}
\usepackage{amsmath}
\usepackage{amssymb}
\renewcommand{\lstlistingname}{Auflistung}

\begin{document}

\pagenumbering{alph}
\begin{titlepage}
  \begin{center}
    \includegraphics[width=\textwidth]{THD-Logo.pdf}
    \vspace{1cm}
    \rule{1\textwidth}{1mm} \\[0.3cm]
    \textsc{\scshape \huge Bachelor Cyber Security}\\
    \rule{1\textwidth}{1mm} \\[2cm]
    {
      \vspace{1cm}
      \Large \textbf{Kryptologie 2}
      \vspace{3cm}
      \Large \textbf{Projektbericht}
    }\\[0.5cm]
    \LARGE \textbf{Ausarbeitung Cryptochallenge: CurveBall}\\[2cm]
    \begin{minipage}[t]{0.4\textwidth}
      \begin{flushleft}
        \normalsize \emph{Autor:}\\[0.3cm]
        Manuel Friedl, Matrikel-Nr.: 1236626\\
        Christof Renner, Matrikel-Nr.: 22301943
      \end{flushleft}
    \end{minipage}
    \begin{minipage}[t]{0.5\textwidth}
      \begin{flushright}
        \normalsize \emph{Betreuer:}\\[0.3cm]
        Prof. Dr. Martin Schramm
      \end{flushright}
    \end{minipage}\\[3cm]
    {\large Deggendorf – \today\\}
  \end{center}
\end{titlepage}

\newpage
\pagenumbering{Roman}

\newpage
\tableofcontents
\newpage

\pagenumbering{arabic}

\section{Einleitung}
Die CurveBall-Challenge ist eine interaktive Lernplattform, die Studierende in die Welt der elliptischen Kurven-Kryptographie einführt. Diese Musterlösungen zeigen die schrittweise Bearbeitung aller Challenges und erklären die zugrundeliegenden kryptographischen Konzepte.


\end{document}