\documentclass{article}
\usepackage[utf8]{inputenc}
\usepackage[hmargin=2.5cm,vmargin=3cm,bindingoffset=0.5cm]{geometry}
\usepackage{graphicx}
\graphicspath{ {figures/} }

\usepackage{listings} 
\renewcommand{\lstlistlistingname}{Auflistungsverzeichnis}
\renewcommand{\lstlistingname}{Auflistung}

\usepackage{todonotes}

\begin{document}

\pagenumbering{alph}  % numbering a, b, c

\begin{titlepage}
	\begin{center}
		
        \includegraphics[width=\textwidth]{THD-Logo.pdf}
	
	    \vspace{1cm}
	
		\rule{1\textwidth}{1mm} \\[0.3cm]

		\textsc{\scshape \huge Bachelor Cyber Security}\\
		
		\rule{1\textwidth}{1mm} \\[2cm] 
		
		{ 
			 \vspace{1cm}
			 
			 \Large \textbf{Modulname}
			 
			 \vspace{3cm}
			 \Large \textbf{Projektbericht}}\\[0.5cm]
			 \LARGE \textbf{CVE-2020-0601 CurveBall Challenge}\\[2cm]
		\begin{minipage}[t]{0.4\textwidth}
			\begin{flushleft} \normalsize
				\emph{Autoren:}\\[0.3cm]
				
			    Vorname Name, Matrikelnummer \#1\\
			    Vorname Name, Matrikelnummer \#2\\
				
			\end{flushleft}
		\end{minipage}
		\begin{minipage}[t]{0.5\textwidth}
			\begin{flushright} \normalsize
				\emph{Betreuer:}\\[0.3cm]
				
    			Vorname Name 
				
			\end{flushright}
		\end{minipage}\\[3cm]
		{\large Ort - \today\\}	
		\vspace{3cm}
	\end{center}
\end{titlepage}

\pagenumbering{Roman} % numbering captial roman

\section*{Eigenständigkeitserklärung}
Hiermit bestätige ich, dass ich die vorliegende Arbeit selbstständig verfasst und keine anderen Publikationen, Vorlagen und Hilfsmitteln als die angegebenen benutzt habe. Alle Teile meiner Arbeit, die wortwörtlich oder dem Sinn nach anderen Werken entnommen sind, wurden unter Angabe der Quelle kenntlich gemacht. Gleiches gilt für von mir verwendete Internetquellen. Die Arbeit ist weder von mir noch von einem/einer Kommilitonen/in bereits in einem anderen Seminar vorgelegt worden. 

\vspace{1cm}

Ort, \underline{\hspace{0.815\textwidth}}\\
\begin{tabular}{lll}
\hspace{3cm} & \small(Datum)\hspace{1cm} & \hspace{2cm}\small(Unterschrift des Studierenden) \normalsize 
\end{tabular}

\newpage

\section*{Challenge-Anleitung: CVE-2020-0601 – CurveBall (Simuliert)}

\subsection{Ziel der Challenge}

Im Rahmen dieser Aufgabe rekonstruieren Sie eine simulierte Angriffsform basierend auf der Schwachstelle \textbf{CVE-2020-0601}, auch bekannt als \textit{CurveBall}. Diese Schwachstelle betrifft die ECC-Zertifikatsprüfung in der Windows CryptoAPI (Crypt32.dll) und erlaubte es Angreifern, gefälschte Zertifikate zu erzeugen, die als vertrauenswürdig eingestuft wurden.

Die Simulation basiert auf einem manipulierten Generatorpunkt auf elliptischen Kurven, bei dem das ursprüngliche Sicherheitsziel – die Überprüfung der Parametrisierungen des öffentlichen Schlüssels – umgangen wird.

\subsection{Challenge-Beschreibung}

Sie erhalten ein vollständig vorkonfiguriertes Docker-Image, das eine simulierte, verwundbare Validierungsumgebung bereitstellt. Ihre Aufgabe ist es, ein manipuliertes ECC-Zertifikat zu erzeugen, das im Kontext dieser simulierten Prüfung als \textbf{gültig} akzeptiert wird.

\subsection{Starten der Umgebung}

Führen Sie folgende Schritte aus:

\begin{enumerate}
    \item Klonen oder extrahieren Sie das bereitgestellte Projektverzeichnis:
    \begin{lstlisting}[language=bash]
    git clone https://github.com/<ihr-repo>/curveball-sim.git
    cd curveball-sim
    \end{lstlisting}

    \item Bauen Sie das Docker-Image:
    \begin{lstlisting}[language=bash]
    docker build -t curveball-challenge .
    \end{lstlisting}

    \item Starten Sie die Container-Umgebung mit:
    \begin{lstlisting}[language=bash]
    docker run --rm -p 443:443 curveball-challenge
    \end{lstlisting}

    \item Nach erfolgreichem Start sehen Sie die simulierte Signaturprüfung im Terminal.
\end{enumerate}

\subsection{Ihre Aufgabe}

\begin{itemize}
    \item Analysieren Sie das Skript \texttt{simulate\_vuln\_check.py}.
    \item Erzeugen Sie mithilfe von \texttt{gen-key.py} eine neue Schlüssel-Paar-Konfiguration, bei der die verwendete elliptische Kurve manipuliert ist (z. B. durch Änderung des Generatorpunkts $G'$).
    \item Signieren Sie eine Nachricht mit dem manipulierten Schlüssel.
    \item Weisen Sie nach, dass die simulierte „verwundbare“ Prüflogik diese Signatur akzeptiert.
\end{itemize}

\subsection{Erfolgskriterium}

Die Aufgabe gilt als gelöst, wenn Ihre Signatur:

\begin{itemize}
    \item \textbf{von der simulierten Prüfroutine als gültig anerkannt wird}, und
    \item \textbf{nicht von der sicheren Referenz-Prüfroutine (optional)} verifiziert werden kann.
\end{itemize}

\subsection{Tipps}

\begin{itemize}
    \item Verwenden Sie für eigene Tests die Python-Bibliothek \texttt{ecdsa}, da sie flexibler im Umgang mit „nichtstandardisierten“ Kurven ist.
    \item Verstehen Sie die Rolle des Generatorpunkts $G$ und warum seine Manipulation die Validierung kompromittieren kann.
    \item Recherchieren Sie die Struktur von X.509-Zertifikaten, falls Sie einen vollständigen Zertifikatspfad nachstellen möchten.
\end{itemize}

\subsection{Optionaler Zusatz}

Erstellen Sie ein Zertifikat (z. B. via \texttt{openssl}), das durch Ihre manipulierte CA signiert wurde und mit der Challenge-Umgebung akzeptiert wird. Ziel ist, die Funktionsweise der Windows CryptoAPI vor der Behebung der Schwachstelle zu simulieren.

\subsection{Abgabe}

Reichen Sie ein kurzes Dokument mit folgenden Inhalten ein:

\begin{enumerate}
    \item Beschreibung Ihres Vorgehens
    \item verwendete Schlüsselparameter (Generatorpunkt, Curve)
    \item Python-Code zur Signaturerzeugung und -verifikation
    \item Screenshot oder Terminal-Output des erfolgreichen Challenge-Durchlaufs
\end{enumerate}

\end{document}
