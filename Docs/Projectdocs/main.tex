\documentclass{article}

\usepackage[utf8]{inputenc}
\usepackage[T1]{fontenc}
\usepackage[ngerman]{babel}
\usepackage[hmargin=2.5cm,vmargin=3cm,bindingoffset=0.5cm]{geometry}

\usepackage{amsmath,amssymb}

\usepackage{graphicx}
\graphicspath{{figures/}}

\usepackage{float}
\usepackage{caption}
\usepackage{subcaption}

\usepackage{tikz}
\usetikzlibrary{arrows.meta, positioning, shapes.symbols}

\usepackage{xcolor,listings}
\definecolor{codegray}{RGB}{245,245,245}
\lstset{
  basicstyle=\ttfamily\small,
  backgroundcolor=\color{codegray},
  frame=single,
  numbers=left,
  numberstyle=\tiny,
  breaklines=true,
  captionpos=b,
  keywordstyle=\color{blue}\bfseries,
  commentstyle=\itshape\color{gray},
  stringstyle=\color{red!60!black},
  language=Python,
  literate=
    {→}{{$\rightarrow$}}2
    {≈}{{$\approx$}}2
    {≤}{{$\leq$}}2
    {≥}{{$\geq$}}2
    {–}{{--}}1
    {—}{{---}}1
}

\usepackage{booktabs}
\usepackage{longtable}
\usepackage{hyperref}
\hypersetup{
  colorlinks=true,
  linkcolor=blue!70!black,
  urlcolor=blue!70!black,
  citecolor=blue!70!black
}

\renewcommand{\lstlistingname}{Auflistung}

\begin{document}

\pagenumbering{alph}
\begin{titlepage}
  \begin{center}
    \includegraphics[width=\textwidth]{THD-Logo.pdf}
    \vspace{1cm}
    \rule{\textwidth}{1mm}\\[0.3cm]
    \textsc{\scshape \huge Bachelor \,--\, Cyber Security}\\
    \rule{\textwidth}{1mm}\\[1.8cm]

    {\Large \bfseries Kryptologie 2}\\[1cm]
    {\Huge \bfseries Projektdokumentation}\\[0.5cm]
    {\Large \bfseries Cryptochallenge: CurveBall (CVE-2020-0601)}\\[2cm]

    \begin{minipage}[t]{0.45\textwidth}
      \begin{flushleft}
        \normalsize \emph{Autoren}\\
        Manuel Friedl – 1236626\\
        Christof Renner – 22301943
      \end{flushleft}
    \end{minipage}
    \hfill
    \begin{minipage}[t]{0.45\textwidth}
      \begin{flushright}
        \normalsize \emph{Betreuer}\\
        Prof.\,Dr.\,Martin Schramm
      \end{flushright}
    \end{minipage}\\[2cm]

    {\large Deggendorf, \today}
  \end{center}
\end{titlepage}

\newpage
\pagenumbering{Roman}
\tableofcontents
\newpage
\pagenumbering{arabic}

\section{Einleitung und Projektkontext}
\subsection{Motivation}
Die Schwachstelle \textbf{CurveBall} (CVE-2020-0601) in der Windows-CryptoAPI
ermöglicht es, X.509-Zertifikate mit manipulierten Elliptic-Curve-Parametern zu
signieren, sodass betroffene Windows-Versionen die Signaturen fälschlich als
gültig akzeptieren.\footnote{Microsoft Security Advisory ADV200002, 14.\,01.\,2020}
Im Modul \emph{Kryptologie 2} fehlte bislang ein modernes
Hands-On-Szenario, um diesen Fehler praktisch zu demonstrieren.

\subsection{Projektziele}
\begin{enumerate}
  \item \textbf{Didaktik}: Vollständiger Angriffszyklus von Discovery bis Exploit.
  \item \textbf{Sicherheit}: Deployment selbst muss trotz absichtlich verletzter Krypto sicher sein.
  \item \textbf{Portabilität}: Schnelle, plattformunabhängige Nutzung via Docker/Podman.
\end{enumerate}

\subsection{Bedrohungsmodell}
\begin{figure}[H]
  \centering
  \begin{tikzpicture}[
      entity/.style={draw, rounded corners, align=center, minimum width=3cm, minimum height=1cm},
      arrow/.style={-Stealth, thick},
      label/.style={font=\small\itshape}
    ]
    \node[entity] (attacker) {Angreifer\\(Fake-CA)};
    \node[entity, right=4cm of attacker] (victim) {Windows 10\\(unpatched)};
    \path[arrow] (attacker) edge[bend left=15] node[label, above]{gefälschtes Zertifikat} (victim);
    \path[arrow] (victim) edge[bend left=15] node[label, below]{akzeptiert} (attacker);
  \end{tikzpicture}
  \caption{Simplifiziertes Threat-Model: fehlende Parameter-Validierung}
  \label{fig:threatmodel}
\end{figure}

\section{Herangehensweise}
\subsection{Zielsetzung}
\begin{itemize}
  \item Demonstration des Angriffs in einer kontrollierten Umgebung.
  \item Vermittlung von DevSecOps-Best-Practices (Linting, CI, Scans).
  \item Bereitstellung als \emph{„One-Click“}-Container, ohne lokale OpenSSL-Konfiguration.
\end{itemize}

\subsection{Methodik}
Wir arbeiteten in zwei Sprints à zwei Wochen.  
Abbildung~\ref{fig:process} zeigt den iterativen Ablauf.

\begin{figure}[H]
  \centering
  \begin{tikzpicture}[
      box/.style={draw, rounded corners, align=center, minimum width=3.2cm, minimum height=1cm},
      >=Stealth
    ]
    \node[box] (analyse) {Bedrohungs-\\Analyse};
    \node[box, right=1.8cm of analyse] (design) {Design \&\\Spezifikation};
    \node[box, right=1.8cm of design] (impl)  {Implementierung};
    \node[box, right=1.8cm of impl] (review) {Code\\Review};
    \node[box, right=1.8cm of review] (deploy) {Docker\\Release};

    \draw[->] (analyse) -- (design);
    \draw[->] (design) -- (impl);
    \draw[->] (impl) -- (review);
    \draw[->] (review) -- (deploy);
  \end{tikzpicture}
  \caption{Iterativer Projektablauf}\label{fig:process}
\end{figure}

\section{Arbeitsaufteilung}

\noindent Die Arbeitsaufteilung erfolgte pragmatisch basierend auf den individuellen Stärken der Teammitglieder, wobei gleichzeitig Wert auf Wissenstransfer und gemeinsames Lernen gelegt wurde.

\begin{longtable}{|p{3cm}|p{5cm}|p{6cm}|}
    \hline
    \textbf{Teammitglied} & \textbf{Hauptaufgaben} & \textbf{Spezifische Aufgaben} \\
    \hline
    \endhead
    
    Christof Renner & Kryptographie \& Frontend & 
    \begin{itemize}
      \item Analyse der CVE-2020-0601 Schwachstelle
      \item Entwicklung der Python-Skripte \texttt{gen\_key.py} und \texttt{simulate\_vuln\_check.py}
      \item Entwicklung des Web-Interface und Zertifikats-Visualizers
      \item Erstellung verständlicher Challenge-Beschreibungen
    \end{itemize} \\
    \hline

    Manuel Friedl & DevOps \& Infrastruktur & 
    \begin{itemize}
      \item Docker-Containerisierung mit Multi-Stage-Builds
      \item CI/CD-Pipeline mit GitHub Actions
      \item Security-Scanning und Linting-Integration
      \item GitHub Container Registry Konfiguration
    \end{itemize} \\
    \hline
    \end{longtable}

\noindent Dabei wurde durch regelmäßige Abstimmungen und gemeinsame Review-Sessions sichergestellt, dass alle Komponenten nahtlos zusammenarbeiten. Die technische Dokumentation wurde gleichmäßig zwischen beiden Teammitgliedern aufgeteilt, wobei jeder etwa 50\% der Dokumentationsarbeit übernahm.

\subsection{Kollaborative Arbeitsweise}

\noindent Trotz der Spezialisierung arbeiteten beide Teammitglieder eng zusammen. Code-Reviews waren Standard, wobei jeder die Arbeit des anderen validierte. Die tägliche Kommunikation erfolgte über Matrix-Chat mit kurzen Standups, ergänzt durch wöchentliche Reviews mit dem Betreuer.

\noindent Das Projektmanagement wurde über GitHub Project Board abgewickelt, was eine transparente Aufgabenverfolgung und effiziente Koordination ermöglichte. Besonders bei komplexen Implementierungen wurden Pair-Programming-Sessions durchgeführt.

\subsection{Herausforderungen und Lösungsansätze}

\noindent Die größten technischen Herausforderungen lagen in der Komplexität der elliptischen Kurven-Mathematik und der Anforderung, sichere Container für absichtlich unsichere Software zu erstellen. Organisatorisch mussten unterschiedliche Stundenpläne koordiniert und Abhängigkeiten zwischen den Arbeitspaketen geschickt gemanagt werden.

\noindent Diese ausgewogene Arbeitsaufteilung führte zu einem erfolgreichen Projektergebnis und einem erheblichen Lernzuwachs für beide Teammitglieder.

\section{Technische Implementierung}



\subsection{Funktionsweise der Vulnerability}
Windows validiert EC-Signaturen ohne sicherzustellen, dass der
öffentliche Schlüssel \(Q\) und der Generator \(G\) wirklich
zur deklarierten Kurve gehören.
Angreifer ersetzen \(G\) durch einen Punkt mit
kleiner Untergruppen-Order, sodass
ECDSA trivial gebrochen wird.

\section{Containerisierung}
\subsection{Dockerfile}
\begin{lstlisting}[language=bash,caption={Auszug aus dem finalen Dockerfile}]
FROM python:3.12-slim AS build
RUN apt-get update && apt-get install -y --no-install-recommends \
    libssl-dev build-essential && rm -rf /var/lib/apt/lists/*
COPY requirements.txt .
RUN pip install --prefix=/install -r requirements.txt

FROM python:3.12-slim
COPY --from=build /install /usr/local
COPY curveball-ctf /app
WORKDIR /app
USER 1001:1001
ENTRYPOINT ["python", "-m", "http.server", "8080"]
\end{lstlisting}

\subsection{Architektur}
\begin{figure}[H]
  \centering
  \begin{tikzpicture}[
      service/.style={draw, rounded corners, minimum width=3.2cm, minimum height=1cm, align=center},
      node distance=2cm
    ]
    \node[service] (browser) {Student\\Browser};
    \node[service, right=of browser] (ctf) {CurveBall-CTF\\Container};
    \node[service, right=of ctf] (registry) {GHCR\\Registry};

    \draw[-Stealth] (browser) -- node[midway, above]{HTTP :8080} (ctf);
    \draw[-Stealth, dashed] (ctf) -- node[midway, above]{Image-Pull} (registry);
  \end{tikzpicture}
  \caption{Container-Deployment in der Lehrumgebung}\label{fig:container}
\end{figure}

\section{CI/CD-Pipeline}
GitHub Actions orchestriert
\emph{Lint\,$\rightarrow$\,Build\,$\rightarrow$\,Test\,$\rightarrow$\,Scan\,$\rightarrow$\,Push}.

\subsection{Workflow-Datei (gekürzt)}
\begin{lstlisting}[language=python,caption={ci.yml – Kernschritte}]
jobs:
  build:
    runs-on: ubuntu-latest
    steps:
      - uses: actions/checkout@v4
      - uses: actions/setup-python@v5
        with: { python-version: "3.12" }

      - name: Ruff Lint
        run: ruff check .

      - name: Build Image
        uses: docker/build-push-action@v5
        with:
          tags: curveball:ci
          push: false

      - name: Trivy Security Scan
        uses: aquasecurity/trivy-action@v0.20.0
        with:
          image-ref: curveball:ci
          severity: HIGH,CRITICAL
\end{lstlisting}

\subsection{Linting‐Tools}
\begin{itemize}
  \item \textbf{ruff}: Python-Lint inkl. Import-Sortierung
  \item \textbf{markdown-lint}: prüft \texttt{README.md}
  \item \textbf{hadolint}: Dockerfile-Best-Practices
\end{itemize}

\section{Nicht umgesetzte VM-Erweiterung}
Ursprünglich war eine vorgefertigte Windows 10-VM (1909, ungepatcht) geplant,
um den Angriff bis zum System-Rootstore zu demonstrieren.
Dies scheiterte aus folgenden Gründen:
\begin{description}
  \item[Licensing] Weitergabe eines vorinstallierten Windows-Images verstößt gegen EULA.
  \item[Storage] 8 GB-Image hätte Git LFS/Kosten gesprengt.
  \item[CI-Runner] GitHub-Actions erlaubt keine Nested-Virtualisation.
\end{description}
Die Container-Variante ist mit rund 280 MB wesentlich leichter verteilbar.

\section{Ausblick}
\begin{enumerate}
  \item \textbf{Windows-Live-Lab}: über Azure Lab Services echte, gepatchte und ungepatchte Hosts anbieten.
  \item \textbf{Automatisierte Angriffskette}: Browser-Automation mit Playwright für einen End-to-End-Exploit.
  \item \textbf{Gamification}: Flag-Server, Leaderboard und Achievements.
\end{enumerate}

\section{Fazit}
Unser Projekt zeigt, wie sich ein kritischer Krypto-Bug in eine
didaktisch wertvolle, aber dennoch sichere Übungsumgebung überführen lässt.
Neben dem technischen Verständnis für CurveBall gewannen Studierende
Einblicke in DevSecOps-Workflows, die heute in jeder Entwicklungsumgebung
zum Standard gehören.

\end{document}
